\documentclass[]{article}

\usepackage{amsmath,amsthm,amssymb}
\DeclareMathOperator{\project}{project}
\DeclareMathOperator{\prox}{prox}
\DeclareMathOperator{\argmax}{argmax}
\DeclareMathOperator*{\minimize}{minimize}
\DeclareMathOperator*{\argmin}{arg\,min}

%\usepackage{program}
\usepackage{algorithm} 
\usepackage{algcompatible}
\usepackage{algpseudocode}

\usepackage[parfill]{parskip} % paragraphs are seperated with an enter space

\usepackage{graphicx}
\graphicspath{{./figs/}}

%opening
\title{A simple introduction to PANOC}
\author{Willem Melis}
\date{\today}

\begin{document}

\maketitle
\begin{abstract}
This is a simple introduction to solving an NMPC problem with the PANOC algorithm. First the proximal operator is introduced with the proximity gradient decent algorithm. Next the link is made between these algorithms and the PANOC algorithm.
\end{abstract}

\newpage
\tableofcontents 
\newpage

\input{part1_MPC.tex}
\chapter{Proximal gradient method}
	\section{Proximal mapping}
		The proximal operator is defined as $\prox_{\gamma g}(x)= \underset{u}{\argmin}(g(u) + \frac{1}{2 \gamma}||u-x||^2_2)$ in \cite{NealParikh}. 
		
		\begin{itemize}
			\item if $h(x)=0$ then $\prox_{\gamma h}(x)=x$ 
			\item if $h(x)=I_c(x)$ where $I_c$ is define in \eqref{eq:indicator function}, the proximity mapping of the indicator function is the orthogonal projection on the set.
		\end{itemize}
		
		The indicator function: is defined in equation ~\eqref{eq:indicator function}.
		\begin{equation}
			I_c = 
			\begin{cases}
			0 & x \in C  \\
			\infty & x \notin C
			\end{cases}
			\label{eq:indicator function}
		\end{equation}
		
		The proximal mapping can be seen as a generalization of a projection. The Appendix contains an other important example, the indicator box function.
	
	\section{Gradient projected method}
		
		\begin{equation}
			\begin{aligned}
			& \underset{x}{\text{argmin}}
			& & f_0(x) \\
			& \text{subject to}
			& & g(x)=0
			\end{aligned}
			\label{eq:prox grad opti problem}
		\end{equation}
		
		The classical gradient decent method cannot be used to solve the problem of \eqref{eq:prox grad opti problem}. As this problem has a condition that must be met by the algorithms solution. If in each iteration the solution is projected on the space spanned by $g(x)=0$ the iteration of \eqref{eq:grad descent} becomes \eqref{eq:projected grad descent}. This algorithm is called the gradient projection method.
		
		\begin{equation}
			x^k = x^{k-1} - \gamma \nabla f(x^{k-1})
			\label{eq:grad descent}
		\end{equation}
		
		\begin{equation}
			x^k = \project_{g(x)=0}[ x^{k-1} - \gamma \nabla f(x^{k-1})]
			\label{eq:projected grad descent}
		\end{equation}
		
		If $g(x)$ is the indicator function of the set onto which is projected. Then  \eqref{eq:projected grad descent} can be written as \eqref{eq:proximal grad descent}, known as the proximal gradient method.
		
		\begin{equation}
				x^k = \prox_{\gamma g}[ x^{k-1} - \gamma \nabla f(x^{k-1})]
			\label{eq:proximal grad descent}
		\end{equation}
	
	\section{The proximal gradient method}
		\eqref{eq:prox grad problem} Can be solved with the proximal gradient method, sometimes called forward backward splitting (FBS) . If the proximal mapping of $g(x)$ is analytically defined. 
			\begin{equation}
			\underset{x}{\argmin} = f(x) + g(x)
			\label{eq:prox grad problem}
			\end{equation}
		
		Inspired by the projected gradient method, the proximal gradient method is defined as \eqref{eq:prox grad method} in \cite{NealParikh}. The $\gamma$ variable is the step size, in order to have a convergence of O(1/k) $\gamma \in(0,1/L)$. If $\gamma \in (1/L,2/L)$ the algorithm will still converge but then its no longer a majorization-minimization method. (more on this see \cite{NealParikh})
		
		\begin{equation}
			x^k = \prox_{\gamma g}\big( x^{(k-1)}- \gamma \nabla f(x^{(k-1)})\big)
			\label{eq:prox grad method}
		\end{equation}	
	
	\section{Proximal minimization algorithm}
		 \cite{QianYang} Contains a short proof that illustrates the proximal mapping as a fixed point minimization algorithm. A property of the conjugated function is used to derive the gradient, see appendix for the theorem.
		 \begin{proof}
		 	The iteration $x^{k+1}=\prox_g(x^k)$ will minimize the smoothed version of function f(x), a shorter version of this proof van be found in \cite{QianYang}. 
		 	\begin{align*}
		 	f_{\mu}
		 	&= \underset{y}{\inf}\Big\{ |y| +\frac{1}{2 \mu}(x-y)^2 \Big\} \\
		 	&=   \frac{1}{2 \mu}||x||^2 + \frac{1}{2\mu} 
		 	\underset{y}{\inf}\Big\{
		 	2 \mu f(y) - 2x^T y + ||y||^2
		 	\Big\} \\
		 	&=  \frac{1}{2 \mu}||x||^2 + \frac{1}{\mu} 
		 	\underset{y}{\sup}\Big\{
		 	x^T y  - \mu f(y) - \frac{1}{2} ||y||^2 \Big \} \\
		 	&= \frac{1}{2\mu }||x||^2 \frac{1}{\mu } \Big( \mu f + \frac{1}{2}||\cdot||^2 \Big)^* (x) \\
		 	\nabla  f_{\mu} 
		 	&= \frac{x}{\mu} - \frac{1}{\mu} \underset{y}{\argmax} 
		 	\Big \{ x^Ty - \mu f(y) - \frac{1}{2}||y||^2 \Big \}\\
		 	& = \frac{1}{\mu}(x - \prox_{\mu f}(x)) \\
		 	\prox_{\mu f}(x)
		 	& = x- \mu \nabla f_{\mu}(x)
		 	\end{align*}
		 	\label{prf:proximal minimiztion alg proof}
		 \end{proof}

		 This means thats the iteration $x^{k+1}=\prox_g(x^k)$ will minimize the smoothed version of function g(x). 

\section{Proximal gradient method with line search}
	\subsection{Starting value gamma }
		\subsubsection{Estimating Lipschitz value}
			The Lipschitz of $\nabla f(x)$ value is a non negative number that complies with \eqref{eq:definition lipschitz value}.
			\begin{equation}
			L = \underset{x \neq y}{\sup} \frac{|\nabla f(y)-\nabla f(x)|}{|y-x|}
			\label{eq:definition lipschitz value}
			\end{equation}
			
			In practice it is not always possible to find the actual Lipschitz value. So the Lipschitz value is estimated ($L'$) locally at the starting location $x_0$ using \eqref{eq:estimated lipschitz value in starting position}. With $\delta=max[\delta_l,10^{-6} \cdot x_0]$ where $\delta_l$ is a small number chosen by the controller designer.
			% in code the safety value is DELTA_LIPSCHITZ_SAFETY_VALUE and set to 10^-6
			
			\begin{equation}
			L' = \frac{|\nabla f(x_0+\delta)-\nabla f(x_0)|}{|\delta|}
			\label{eq:estimated lipschitz value in starting position}
			\end{equation}
			
			The Lipschitz value is not explicitly saved but is used to estimate the line-search parameter $\gamma$. The backtracking of the proximal gradient algorithm then further improves $\gamma$. As the algorithm progresses and $\gamma$ improves, so does the estimation of the Lipschitz value indirectly.
		
		\subsubsection{Estimating gamma}	
			\cite{LorenzoStella2017} states that $\gamma<\frac{1}{L}$ guarantees convergence to a local minimum. As gamma needs to be smaller than $\frac{1}{L}$ a safety value is introduced. This idea was copied over from the kul-forbes/ForBES library by Lorenzo Stella and Panos Patrinos, which uses a $\beta$ of 0.05. And leads to \eqref{eq:starting value gamma}.
			\begin{equation}
			\gamma = \frac{1-\beta}{L}
			\label{eq:starting value gamma}
			\end{equation}		
	
	\subsection{Backtracking gamma}			
		\subsubsection{Backtracking in proximal gradient descent used in FBS}
			The line-search makes use of the quadratic bound of \eqref{eq:Armijo's sufficient decrease condition prox grad PANOC} Which is an quadratic bound, by backtracking on $\gamma$ the optimal step size is retrieved.
			
			\begin{equation}
			f({\bar{x}}) \leq f(x) - \nabla f(x)^T[x-\bar{x}] + \frac{1}{2 \gamma}||x-\bar{x}||^2
			\label{eq:Armijo's sufficient decrease condition prox grad PANOC}
			\end{equation}
			
			This leads to algorithm~\ref{alg:backtracking on gamma} the proximal gradient method. \eqref{eq:Armijo's sufficient decrease condition prox grad PANOC} can be seen as a quadratic model, as the Lipschitz value of the gradient is equal to $L=\frac{1}{\gamma}$.
			
			\begin{algorithm}
				\caption{backtracking $\gamma$}
				\label{alg:backtracking on gamma}
				\begin{algorithmic}[1]
					\Procedure {lineasearch\_gamma}{x,$\gamma$}
					\State $\bar{x} = \prox_{\gamma g}\big( x- \gamma \nabla f(x)\big)$
					\While{$f({\bar{x}}) > f(x) - \nabla f(x)^T[x-\bar{x}] + \frac{1}{2 \gamma}||x-\bar{x}||^2$}
					\State $\gamma = \frac{\gamma}{2}$
					\State $\bar{x} = \prox_{\gamma g}\big( x- \gamma \nabla f(x)\big)$
					\EndWhile
					\State \Return $\gamma$
					\EndProcedure
				\end{algorithmic}
			\end{algorithm}
	\subsection{Final algorthm}
		The final algorithm~\ref{alg:proximal gradient PANOC with backtracking} delivers the upward direction. $x_{k+1}=x_k - direction$.
		\begin{algorithm}
			\caption{proximal gradient PANOC with backtracking}
			\label{alg:proximal gradient PANOC with backtracking}
			\begin{algorithmic}[1]
				\Procedure{get\_proximal\_gradient\_step}{x,$\gamma$}
				\State $\gamma$=LINESEARCH\_GAMMA($\gamma$)
				\State $\bar{x} = \prox_{\gamma g}\big( x- \gamma \nabla f(x)\big)$
				\State \Return direction=$[x-\bar{x}]$, $\gamma$
				\EndProcedure
			\end{algorithmic}
		\end{algorithm}
\section{Proximal gradient alternative view}
	\subsection{Majorization-minimization algorithm}
	A Majorization-minimization algorithm is a type of algorithm that minimizes a surrogate function. The surrogate function is a approximation of the actual problem. And needs to have the conditions described in \eqref{eq:MM algorithm conditions}, where h(x,y) is a surrogate function and f(x) is the actual problem. The problem described in \eqref{eq:MM algorithm formula step} is iteratively solved until convergence.
	\begin{equation}
		\begin{cases}
			h(x,x) = f(x) \\
			h(x,y) \leq f(x)
		\end{cases}
		\label{eq:MM algorithm conditions}
	\end{equation}
	
	\begin{equation}
		x_{k+1} = \argmin_y h(x_k,y)
		\label{eq:MM algorithm formula step}
	\end{equation}

	\subsection{The proximal gradient as majorization-minimization algorithm}
	\cite{NealParikh} States that the proximal gradient algorithm can be seen as a majorization-minimization algorithm. Its surrogate function is described in \eqref{eq:surrogate function}, proof~\ref{prf:proximal gradient as MM} illustrates how minimizing the surrogates function is equal as taking a step of the proximal gradient algorithm.
	
	\begin{equation}
		h(x,y) = f(x) + \nabla f(x)^T(x-y) + \frac{1}{2 \cdot \gamma}||x-y||^2_2
		\label{eq:surrogate function}
	\end{equation}
	
	\begin{proof}
		$h(x,y) = f(x) + \nabla f(x)^T(x-y) + \frac{1}{2 \cdot \gamma}||x-y||^2_2$ is the MM surugate function of proximal gradient algorithm. The gradient is set too zero and solved for y.
		\begin{align*}
		h(x,y)
		& = f(x) + \nabla f(x)^T(x-y) + \frac{1}{2 \cdot \gamma}||x-y||^2_2 \\
		\frac{\partial h(x,y)}{\partial y}
		& = 0 + \nabla f(x) + \frac{1}{2 \cdot \gamma} + \frac{1}{2 \cdot \gamma}\frac{\partial ||x-y||^2_2}{\partial y}  \\	
		\frac{\partial ||x-y||^2_2}{\partial y}
		& = \frac{\partial(x-y)^T(x-y)}{\partial y} = \frac{\partial(-2x^Ty + y^Ty)}{\partial y} = -2x+y \\
		\frac{\partial h(x,y)}{\partial y}
		& = \nabla f(x) - \frac{x}{\gamma} + \frac{y}{2 \gamma} = 0 \\
		0
		& = \gamma \nabla f(x) - x + \frac{y}{2} \\
		y & = \gamma \nabla f(x) - x
		\end{align*}
		\label{prf:proximal gradient as MM}
	\end{proof}
\clearpage
\chapter{PANOC algorithm}
	This section is based on \cite{LorenzoStella2017} and \cite{AjaySathya2017}, the difference is that this text is focused implementation. And so the formula's often look slightly different. The Forward backward envelop part is based on \cite{Themelis}, which contains a slightly different algorithm that also uses the Forward backward envelop.
	\section{Introducing PANOC}
		The PANOC algorithm is an accelerated version of the proximal gradient descent algorithm. The direction  $x-\bar{x}$ is a convex combination of the proximal gradient algorithm's direction and an accelerator that makes use of curvature information. It is formally defined in \cite{LorenzoStella2017} as \eqref{eq:weighting linesearch FBE}, with $\tau_k \leq 1$.
		
		\begin{equation}
			x_{k+1} = x_k + (1-\tau_k)\cdot (x-\bar{x}) + \tau_k d_k
			\label{eq:weighting linesearch FBE}
		\end{equation}
		
		The term $\tau_kd_k$ can accelerate the convergence if $\tau_k\neq0$. The step $d_k$ is calculated using a quasi-newton algorithm. As a quasi-newton method uses curvature information of the cost function, it uses information not available to a gradient descent based method. 
		
		Furthermore it has a super linear convergence rate when it gets close to the solution. Which is much faster then the sub-linear convergence of the gradient descent algorithm . On top of that, the quasi-newton method does not require additional function or gradient evaluations. Which are the major contributors to the computational costs of the algorithm.
		
	\section{Quasi newton method}
		\subsubsection{Problem definition}
			The residue of the iteration \eqref{eq:prox grad method} can be used to solve the optimization problem.  By setting the residue in \eqref{eq:residue prox grad method} to zero, a local optimum can be obtained.
			
			\begin{equation}
			R_{\gamma}(x)= \frac{1}{\gamma}\left[ x - \prox_g( x - \nabla f(x)\gamma) \right]
			\label{eq:residue prox grad method}
			\end{equation}
			
			The root of \eqref{eq:residue prox grad method} can be obtained using the Newton iteration of \eqref{eq:newton iteration residual}. Where $H_k$ satisfies the inverse secant condition of \eqref{eq:newton iteration residual inverse secant}. If the implementation is aimed at embedded software L-BFGS is an excellent choice to solve \eqref{eq:residue prox grad method}. As it doesn't explicitly save the Hessian matrix.
			
			\begin{equation}
			x^{k+1} = x^k -H_kR_{\gamma}(x^k)
			\label{eq:newton iteration residual}
			\end{equation}
			\begin{equation}
			x^{k+1} - x^k = H_{K+1} \Big( R_{\gamma}(x^{k+1})- R_{\gamma}(x^k) \Big)
			\label{eq:newton iteration residual inverse secant}
			\end{equation}
		
	\section{Forward backward envelop}	
		Newton iterations only converge when they are close enough to the solution. In order to get good global behavior, a proper global strategy is required. The optimization problem is changed from $\varphi(x) = f(x) + g(x)$ to \eqref{eq:FBE definition using Moreau envelope}. This problem is smoother while it still has the same optimal solution.(proof see \cite{LorenzoStella2017} and \cite{Themelis}) The same $\gamma$ as with the proximal gradient should be used, notice how the FBE contains the line-search condition used in the proximal gradient. (more on the FBE in \cite{Themelis})
		
		The Moreau envelope is de defined as \eqref{eq:Moreau envelope}, this smooths a function. Using simple algebra \eqref{eq:FBE definition using Moreau envelope} can be transformed into \eqref{eq:FBE definition using quadratic approximation}. The solution $y$ of the infimum in \eqref{eq:FBE definition using quadratic approximation} is $\bar{x}$. Considering the close relationship between the Moreau envelope and the proximal operator this is to be expected. (see more in \cite{Themelis}) 
		
		An alternative way to look at the FBE is illustrated in \cite{AjaySathya2017}, where the problem can be seen as minimizing a quadratic approximation  $f(x) +  \nabla f(x)^T(y-x) + g(y) + \frac{1}{2 \gamma} ||x-y||^2  $ towards y in point x.  Because $L = \frac{1}{\gamma}$, with L as the Lipschitz constant of the gradient.
		
		\begin{equation}
			g^{\gamma} = \underset{y}{\inf} \big \{f(y)+\frac{1}{2 \cdot \gamma}||x-y||^2 \big \}
			\label{eq:Moreau envelope}
		\end{equation}
		
		\begin{equation}
		\varphi_{\gamma} = f(x) - \frac{\gamma}{2}||\nabla f(x)||^2 + g^{\gamma} \big(x-\gamma \nabla f(x) \big)
		\label{eq:FBE definition using Moreau envelope}
		\end{equation}
		
		\begin{equation}
		\varphi_{\gamma} =   f(x) + \underset{y}{\inf} \Big\{ \nabla f(x)^T(y-x) + g(y) + \frac{1}{2 \gamma} ||x-y||^2  \Big\}
		\label{eq:FBE definition using quadratic approximation}
		\end{equation}
		
		\begin{proof}
			The solution to the infimum of \eqref{eq:FBE definition using quadratic approximation} is y=$\bar{x}=\prox_g(x-\gamma \nabla f(x))$
			\begin{align*}
			\prox_g(\bar{x}) 
			&=\prox_g(x- \gamma \nabla f(x)) \\
			&= \underset{y}{\argmin} \Big \{ g(y) 
			+ \frac{1}{2 \gamma}||(y-x) + \gamma \nabla f(x)||^2 \Big \} \\
			&= \underset{y}{\argmin} \Big \{ g(y) 
			+ \frac{1}{2 \gamma} \big[||y-x||^2 + 2 \gamma \nabla f(x)^T(y-x) + ||\nabla f(x)||^2\gamma^2 \big] \Big \} \\
			&= \underset{y}{\argmin} \Big \{ g(y) 
			+ \frac{1}{2 \gamma} \big[||y-x||^2 + 2 \gamma \nabla f(x)^T(y-x)  \big] \Big \}\\
			&= \underset{y}{\argmin} \Big \{   \nabla f(x)^T(y-x)  + g(y) 
			+ \frac{1}{2 \gamma} ||y-x||^2  \Big  \}
			\end{align*}
			\label{prf:prox is solution to FBE inf}
		\end{proof}
		
		\eqref{eq:practical implementation of FBE} Is the practical implementation of the FBE. The parameter gamma is the line-search parameter used in the proximal gradient descent. The first 3 terms are the same as with the line-search on $\gamma$. The last term $g(\bar{x})$ is new and ensures that the solution complies with the constraint.
		
%		\cite{AjaySathya2017} has an excellent example to illustrate this.(Figure~\ref{fig:FBE illustration}).
%		
		\begin{equation}
			\begin{aligned}	
				& \varphi(\gamma,x)= 
				&& f(x) - \nabla f(x)^T(x-\bar{x}) + \frac{1}{2 \gamma}||x-\bar{x}||^2  + g(\bar{x})
				\\
				& with 
				&&\bar{x} = \prox_g( x - \gamma\nabla f(x)) 
			\end{aligned} 
			\label{eq:practical implementation of FBE}
		\end{equation}
		
%		\begin{figure}[H]
%			\centering
%			\label{fig:FBE illustration}
%			\includegraphics[width=0.6\textwidth]{FBE}
%			\caption{FBE example copied over from \cite{AjaySathya2017}, red is?? blue is ?? TODO expain this figure}
%		\end{figure}
	
	\section{Line-search with FBE}
	In \cite{LorenzoStella2017} the line-search condition is specified as equation ~\eqref{eq:line-search with FBE}. The line-search parameter $\tau$ determines $x^{k+1}$ by weighting the convex combination of the proximal gradient step and the L-BGFS step as defined in \eqref{eq:linea-search tau definition}.  \cite{LorenzoStella2017} Specifies that $\sigma \in (0, \gamma \frac{1-\gamma\cdot L}{2})$. As stated before it is assumed that $L=\frac{1-\beta}{\gamma}$, some simple algebra will lead to the condition $\sigma \in (0,\frac{\beta \gamma}{2})$.
	
	 \eqref{eq:practical line-search with FBE} Is the practical implementation of \eqref{eq:line-search with FBE}, it incorporates the safety value $\beta$. A new constant $\alpha \in (0,1)$ is introduced, a possible value for $\alpha$ would be 0.5 ($\alpha=0.5$ is the choice used by the Matlab implementation of PANOC in ForBes known as zerofpr2).
	
	\begin{equation}
		x^{k+1} = u_k - (1-\tau_k)\cdot (x-\bar{x}) + \tau \cdot dir_{LBFGS}
		\label{eq:linea-search tau definition}
	\end{equation}
	
	\begin{eqnarray}
		\label{eq:line-search with FBE}
		\varphi_{\gamma}(x^{k+1})\leq\varphi_{\gamma}(x^{k}) - \sigma ||\frac{x-\bar{x}}{\gamma}||^2 \\
		=
		\varphi_{\gamma}(x^{k}) - \frac{\sigma}{\gamma^2} ||x-\bar{x}||^2
	\end{eqnarray}
	
	\begin{equation}
		\varphi_{\gamma}(x^{k+1}) \leq 		\varphi_{\gamma}(x^{k}) - \alpha \frac{\beta }{\gamma \cdot 2} ||x-\bar{x}||^2
		\label{eq:practical line-search with FBE}
	\end{equation}
	
	
		
		\begin{algorithm}
			\caption{PANOC}
			\label{alg:PANOC}
			\begin{algorithmic}[1]
				\Procedure {PANOC\_GET\_NEW\_LOCATION}{$x^k$,$\gamma$}
				\State [$(x-\bar{x})$ , $\gamma$] = GET\_PROXIMAL\_GRADIENT\_STEP($\gamma$,$x^k$)
				\State $ dir_{LBFGS}$ = LBFGS($x^k$)
				\State $\tau =1$
				\State $x^{k+1} = x_k - (1-\tau_k)\cdot (x-\bar{x}) + \tau \cdot dir_{LBFGS}$
				\While{$\varphi_{\gamma}(x^{k+1}) > 		\varphi_{\gamma}(x^{k}) - \alpha \frac{\beta}{\gamma \cdot 2} ||(x-\bar{x})||^2$}
				\State $\tau = \tau / 2$
				\State $x^{k+1} = x_k - (1-\tau_k)\cdot (x-\bar{x}) + \tau \cdot dir_{LBFGS}$
				\EndWhile
				\EndProcedure
			\end{algorithmic}
		\end{algorithm}
	

\clearpage

\section{Implementation}
	\subsection{the dynamic C code}
	\subsection{The static C lib}
		The static c code is structured in an layered architecture. The PANOC algorithm is the first layer. PANOC calls the second layer containing the LBFGS and proximal gradient descent. And finally the lowest layer containing the Casadi functions, a buffer and the Lipschitz estimator.
		\begin{figure}[H]
			\centering
			\label{fig:software architecture}
			\includegraphics[width=0.8\textwidth]{visio_software_arch}
			\caption{software architecture}
		\end{figure}
	
\clearpage
\appendix 
\section{Function Definitions}
	\subsection{box function}
		1 dimension:
		\begin{equation}
			box_{1D}(u) =
			\begin{aligned}
			 & 1 && u \in [-U_b:U_b]\\
			 & 0 && otherwise
			\end{aligned}
			\label{eq:box function 1 dimension}
		\end{equation}
		N dimensions:
		\begin{equation}
		box(u) = min\left[ \sum_{k=1}^ N box_{1D}(u) \right]
		\label{eq:box function N dimensions}
		\end{equation}
	\subsection{Indicator Box function}
		\begin{equation}
			I[box]=
			\begin{aligned}
			& 0 && u \in [-U_b:U_b]\\
			& \inf && otherwise
			\end{aligned}
		\end{equation}
		
		\begin{equation}
		prox[I[box]]=
		\begin{aligned}
		& u && u \in [-U_b:U_b]\\
		& -U_b && u \in [-\inf:-U_b]\\
		& -U_b && u \in [U_b:\inf]\\
		\end{aligned}
		\end{equation}
\section{functions}

	\subsection{Conjugate of strongly convex function}
		Following lemma's are useful from 
		\begin{equation}
			f^*(x)= \underset{u \in dom(f)}{<y,x>-f(y)}
		\end{equation}
		
		If $\nabla f^*$ is lipschitz and obeys \eqref{eq:appendix f lip}, then $f^*$ is well defined and differentiable. (assume dom(f) is convex and closed)
		\begin{equation}
			\nabla f^*(x) = y^* = \argmax <y,x> - f(y)
		\end{equation}
		
		\begin{equation}
			|| \nabla f^*(x) - \nabla f^*(y) ||_2 \leq \mu^{-1} ||x-y||_2
			\label{eq:appendix f lip}
		\end{equation}
\clearpage
\section{Proof FBE alternate equation}

\begin{proof}
	$\varphi_{\gamma} =   f(x) + \underset{y}{\inf} \Big\{ \nabla f(x)^T(y-x) + g(y) + \frac{1}{2 \gamma} ||x-y||^2  \Big\} $
	\begin{align*}
	g^{\gamma} 	&=  \underset{y}{\inf} \big \{f(y)+\frac{1}{2 \cdot \gamma}||x-y||^2 \big \} \\
	\varphi_{\gamma} 
	&= f(x) - \frac{\gamma}{2}||\nabla f(x)||^2 + g^{\gamma} \big(x-\gamma \nabla f(x) \big) \\
	&= f(x) - \frac{\gamma}{2}||\nabla f(x)||^2 + g^{\gamma} \big(\bar{x} \big)\\
	g^{\gamma} (\bar{x})
	&=\underset{y}{\inf} \Big\{g(y)+\frac{1}{2 \gamma}||\bar{x}-y||^2 \Big\}	\\
	\frac{1}{2 \gamma}||\bar{x}-y||^2
	&=\frac{1}{2 \gamma} \Big [ (\bar{x}-y)^T(\bar{x}-y) \Big]\\
	&=\frac{1}{2 \gamma} \Big [ x^Tx - 2 x^Ty + y^Ty \Big]\\
	\bar{x}^T\bar{x}
	&=[x- \gamma \nabla f(x)]^T[x- \gamma \nabla f(x)] \\
	&= x^Tx -2x^T\nabla f(x) \gamma + \gamma^2 \nabla f(x)^T\nabla f(x)-2x^Ty \\
	&=-2(x-\gamma f\nabla(x))^Ty\\
	&=-2x^Ty + 2\gamma \nabla f(x)^Ty \\
	\frac{1}{2 \gamma}||\bar{x}-y||^2 
	& =\frac{1}{2 \gamma}[x^Tx-2x^T\nabla f(x) \gamma + \gamma^2 \nabla f(x)^T\nabla f(x) -2x^Ty + 2\gamma \nabla f(x)^Ty +y^Ty] \\
	& = \frac{1}{2 \gamma}[-2x^T\nabla f(x) \gamma  + 2\gamma \nabla f(x)^Ty + \gamma^2 \nabla f(x)^T\nabla f(x) +x^Tx -2x^Ty +y^Ty]\\
	&= \frac{1}{2 \gamma}[ 2\gamma \nabla f(x)^T(y-x) + \gamma^2||\nabla f(x)||^2 + (x-y)^T(x-y)]\\
	&= \frac{1}{2 \gamma}[ 2\gamma \nabla f(x)^T(y-x) + \gamma^2||\nabla f(x)||^2 + ||x-y||^2] \\
	&=  \nabla f(x)^T(y-x) +\frac{\gamma}{2}||\nabla f(x)||^2 + \frac{1}{2 \gamma} ||x-y||^2 \\
	g^{\gamma} (\bar{x})
	&=\underset{y}{\inf} \Big\{g(y)+ \nabla f(x)^T(y-x) +\frac{\gamma}{2}||\nabla f(x)||^2 + \frac{1}{2 \gamma} ||x-y||^2  \Big\} \\
	&= \frac{\gamma}{2}||\nabla f(x)||^2 + \underset{y}{\inf} \Big\{g(y)+ \nabla f(x)^T(y-x) + \frac{1}{2 \gamma} ||x-y||^2  \Big\} \\
	\varphi_{\gamma} 
	&= f(x) - \frac{\gamma}{2}||\nabla f(x)||^2 + g^{\gamma} \big(\bar{x} \big)\\
	&= f(x) - \frac{\gamma}{2}||\nabla f(x)||^2 +  \frac{\gamma}{2}||\nabla f(x)||^2 + \underset{y}{\inf} \Big\{g(y)+ \nabla f(x)^T(y-x) + \frac{1}{2 \gamma} ||x-y||^2  \Big\}\\
	&=   f(x) + \underset{y}{\inf} \Big\{ \nabla f(x)^T(y-x) + g(y) + \frac{1}{2 \gamma} ||x-y||^2  \Big\} 
	\end{align*}
	\label{prf:}
\end{proof}



$$  $$
$$  $$



$$ $$

\newpage
\bibliography{mylib}
\bibliographystyle{plain}

\end{document}
