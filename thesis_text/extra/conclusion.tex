\chapter{Conclusion}
%\label{cha:conclusion}
The final chapter contains the overall conclusion. It also contains
suggestions for future work and industrial applications.

The nmpc-codegen library is implemented in Matlab and Python, simple model predictive control problems works very well. As illustrated in the text. It requires absolutely no knowledge of how the solver works. But only requires control theory knowledge to generate the controller. The generate controller, has a very simple interface that requires nearly no control theory knowledge. So that the user of the generated library, does necessarily have to be a control engineer but could be an embedded software engineer.

However as the obstacles get more complex, the problem becomes more complex. And the direct single shot approach results in solutions that are not viable. The algorithm gets stuck on a local minimum, which might be in an nonviable solution.

The last chapter tries to partially solve this by using a Lagrangian. This is ongoing research by Ben Hermans. Its implementation is rather limited as it is ongoing research.

%%% Local Variables: 
%%% mode: latex
%%% TeX-master: "thesis"
%%% End: 
