\chapter{Conclusion}
%\label{cha:conclusion}
%The final chapter contains the overall conclusion. It also contains
%suggestions for future work and industrial applications.

\section{Nmpc-codegen library}
The nmpc-codegen library is implemented in Matlab and Python, which makes it quiet straight forward to design and test a controller. It requires absolutely no knowledge of how the solver works. But only requires control theory knowledge to generate the controller.

The generated controller, has a very simple interface that requires nearly no control theory knowledge. So that the user of the generated C library, does necessarily has to be a control engineer but could be an embedded software engineer. Who has no idea how a MPC controller works.

However as the problem becomes more complex. The algorithm can get stuck on a local minimum, which might be in an nonviable solution. Or it might violate the constraint by a to large margin. This can partially be avoided by properly tuning the weights of the constraints. 

However this weight tuning is rather hard, as increasing the weights worsens the condition of the problem. So the controller designer often ends up with an impossible choice. The last chapter tries to partially solve this by using a Lagrangian. And succeeds for some constraints, it's easier to tune and improves the performance of the controller. 

\section{Future work}
The PANOC algorithm does not require more advanced algebra operations, like solving a system. Only simple vector additions and inner product are required. This makes it quiet easy to implement in hardware. As it's very straightforward to implement with flip-flops and adders. The only hard part is the cost function and it's gradient, as Casadi does not support VHDL or Verilog at this time.


%%% Local Variables: 
%%% mode: latex
%%% TeX-master: "thesis"
%%% End: 
