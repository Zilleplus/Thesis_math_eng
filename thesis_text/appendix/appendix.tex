\chapter{Abstract Dutch}
Moderne controle systemen maken vaak gebruik van "model predictive control" algoritmes. Wanneer lineaire modellen gebruikt worden, moet een convex optimalisatie probleem opgelost worden. De algoritmes voor het oplossen van convexe optimalisatie problemen zijn uitgebreid onderzocht en ge\"{i}mplementeerd. 

Wanneer een niet-lineaire model wordt gebruikt, moet een niet-lineaire niet-convex optimalisatie probleem worden opgelost. Dit probleem kan worden opgelost met iteratieve algoritmes zoals "interior point" of SQP algoritmes. Deze algoritmes gebruiken veel geheugen, wat ze vaak ongeschikt maken voor embedded toepassingen. Het "proximal gradient" algoritme gebruikt weinig geheugen maar heeft slechts sub-lineaire convergentie. Dit is waar het PANOC algoritme interessant wordt, het kan namelijk een niet-lineaire niet-convexe optimalisatie problemen oplossen met veel minder geheugen dan een "interior point" algoritme, en met alsook super-lineaire convergentie. 

Het doel van deze masterproef is een raamwerk te implementeren die een niet-lineaire MPC controller genereert in C. De controller maakt gebruik van het PANOC algoritme voor het oplossen van het optimalisatie probleem. De code is specifiek gericht aan embedded toestellen, waardoor PANOC een uitstekende keuze is. De gebruiker van de software wordt verondersteld alleen controletheorie te kennen, en hoeft niets over PANOC algoritme of embedded software te kennen.

De implementatie is gerealiseerd in Matlab en Python, aangezien dit de twee dominerende talen zijn bij computer simulaties van controlesystemen. Er wordt geen gebruik gemaakt van externe bibliotheken om de C code te genereren met uitzondering van Casadi. Er wordt gebruik gemaakt van het pakket Casadi om de gradi\"{e}nt te berekenen. Het PANOC algoritme is een convexe combinatie van het "proximal gradient" algoritme en L-BFGS.  Het is mogelijk om vanuit Matlab of Python simulaties uit te voeren op de gegenereerde C code. Zo kunnen de nodige controle parameters afgesteld worden vanuit Matlab of Python.

De gegenereerde C code is getest met de GNU C compiler, Clang LLVM compiler, Microsoft C compiler en de Intel C compiler onder de ANSI C89 standaard. Er worden geen externe C bibliotheken gebruikt zodat de code eenvoudig kan gecompileerd worden voor verschillende platformen. Het enige noodzakelijke is een C compiler.

Eerst wordt MPC, PANOC algoritme en de software architectuur theoretisch besproken. Vervolgens wordt aangetoond dat PANOC bij eenvoudige systemen sneller is dan "interior point" methodes, maar bij complexe systemen verliest het zijn snelheids voordeel. PANOC verbruikt wel altijd minder geheugen.

Tot slot blijkt uit recentelijk onderzoek dat het gebruik van de "augmented Lagrangian" kan leiden tot betere resultaten. Het toevoegen van voorwaarden op het controlesysteem kan namelijk leiden tot een slecht geconditioneerd probleem. De "augmented Lagrangian" methode bied hiervoor soms een oplossing. Door het probleem in verschillende stappen op te lossen kan er soms een beter resultaat bereikt worden.

\chapter{Mathematical definitions}
	\section{Box function}
	\label{appendix:box function}
		1 dimension:
		\begin{equation}
			\boxfunction_{1D}(u) =
			\begin{cases}
				 1 & u \in [-U_b:U_b]\\
				 0 & \otherwise
			\end{cases}
			\label{eq:box function 1 dimension}
		\end{equation}
		N dimensions:
		\begin{equation}
			\boxfunction(u) = \min\left[ \sum_{k=1}^ N \boxfunction_{1D}(u) \right]
			\label{eq:box function N dimensions}
		\end{equation}
	\section{Indicator Box function}
	\label{appendix:indicator box function}
		\begin{equation}
			I_{\boxfunction}(u)=
			\begin{cases}
				0 & \underline{u}\leq u \leq \overline{u}\\
				\inf & \otherwise
			\end{cases}
		\end{equation}
		
		\begin{equation}
			\prox_{\gamma I_{\boxfunction}}(u)=
			\begin{cases}
				u & \underline{u}\leq u \leq \overline{u}\\
				\underline{u} & -\inf \leq u \leq \underline{u}\\
				\overline{u} & \overline{u}\leq u \leq \inf\\
			\end{cases}
		\end{equation}
	
		
\chapter{Benchmarks with trailer model}
\label{appendix:paths trailer simulations}

\begin{figure}[H]
	\centering
	\begin{subfigure}[b]{0.40\textwidth}
		\centering
		\includegraphics[width=1.2\textwidth]{demos/demo1}
		\caption{demo 1}
		\label{fig:demo 1}
	\end{subfigure}
	\hfill
	\begin{subfigure}[b]{0.40\textwidth}
		\centering
		\includegraphics[width=1.2\textwidth]{demos/demo2}
		\caption{demo 2}
		\label{fig:demo 2}
	\end{subfigure}
	\begin{subfigure}[b]{0.40\textwidth}
		\centering
		\includegraphics[width=1.2\textwidth]{demos/demo3}
		\caption{demo 3}
		\label{fig:demo 3}
	\end{subfigure}
	\hfill
	\begin{subfigure}[b]{0.40\textwidth}
		\centering
		\includegraphics[width=1.2\textwidth]{demos/demo4}
		\caption{demo 4}
		\label{fig:demo 4}
	\end{subfigure}
	\caption{Path of demo's}
	\label{fig:demos}
\end{figure}
\begin{figure}[H]
	\centering
	\begin{subfigure}[b]{0.45\textwidth}
		\centering
		\includegraphics[width=1.2\textwidth]{demos/demo5}
		\caption{demo 5}
		\label{fig:demo 5}
	\end{subfigure}
\end{figure}

\section{Simulations without state noise}
\label{appendix:benchmarks trailer without noise}

\begin{table}[H]
	\centering
	\begin{tabular}{|l|c|c|c|c|}
		\hline
		&\textbf{demo5}&\textbf{demo2}&\textbf{demo3}\\\hline
		\textbf{nmpc-codegen (PANOC)}&9.00e-01&1.00e-01&4.10e-01\\\hline
		\textbf{ForBeS zerofrp2 (PANOC)}&2.99e+01&4.20e+00&5.56e+00\\\hline
		\textbf{PANOC draft}&5.73e+00&1.29e+00&2.00e+00\\\hline
		\textbf{fmincon:interior-point}&8.62e+01&2.31e+01&2.03e+01\\\hline
		\textbf{fmincon:sqp}&2.18e+01&1.15e+01&9.29e+00\\\hline
		\textbf{fmincon:active-set}&8.37e+01&2.36e+01&2.08e+01\\\hline
		\textbf{OPTI:ipopt}&3.79e+01&7.08e+00&8.21e+00\\\hline
	\end{tabular}
	\caption{Mean time till convergence in milliseconds}
	\label{tbl:mean time till convergence}
\end{table}

\begin{table}[H]
	\centering
	\begin{tabular}{|l|c|c|c|c|}
		\hline
		&\textbf{demo5}&\textbf{demo2}&\textbf{demo3}\\\hline
		\textbf{nmpc-codegen (PANOC)}&100&100&100\\\hline
		\textbf{ForBeS zerofrp2 (PANOC)}&3323&4202&1357\\\hline
		\textbf{PANOC draft}&637&1289&489\\\hline
		\textbf{fmincon:interior-point}&9579&23107&4943\\\hline
		\textbf{fmincon:sqp}&2422&11469&2265\\\hline
		\textbf{fmincon:active-set}&9303&23578&5084\\\hline
		\textbf{OPTI:ipopt}&4213&7083&2003\\\hline
	\end{tabular}
	\caption{Mean relative time till convergence in milliseconds}
	\label{tbl:mean relative time till convergence}
\end{table}

\begin{table}[H]
	\centering
	\begin{tabular}{|l|c|c|c|c|}
		\hline
		&\textbf{demo5}&\textbf{demo2}&\textbf{demo3}\\\hline
		\textbf{nmpc-codegen (PANOC)}&2.20e+01&8.00e+00&4.00e+01\\\hline
		\textbf{ForBeS zerofrp2 (PANOC)}&7.95e+02&1.25e+02&3.06e+02\\\hline
		\textbf{PANOC draft}&1.12e+02&3.66e+01&1.30e+02\\\hline
		\textbf{fmincon:interior-point}&1.55e+03&2.29e+02&2.95e+02\\\hline
		\textbf{fmincon:sqp}&1.01e+02&8.83e+01&1.10e+02\\\hline
		\textbf{fmincon:active-set}&7.83e+02&2.70e+02&2.45e+02\\\hline
		\textbf{OPTI:ipopt}&2.73e+02&4.65e+01&7.65e+01\\\hline
	\end{tabular}
	\caption{Max time till convergence in milliseconds}
	\label{tbl:max time till convergence}
\end{table}

\begin{table}[H]
	\centering
	\begin{tabular}{|l|	c|c|c|c|}
		\hline
		&\textbf{demo5}&\textbf{demo2}&\textbf{demo3}\\\hline
		\textbf{nmpc-codegen (PANOC)}&0.00e+00&0.00e+00&0.00e+00\\\hline
		\textbf{ForBeS zerofrp2 (PANOC)}&2.11e+00&2.03e+00&2.01e+00\\\hline
		\textbf{PANOC draft}&5.16e-01&5.26e-01&5.16e-01\\\hline
		\textbf{fmincon:interior-point}&8.95e+00&8.35e+00&7.57e+00\\\hline
		\textbf{fmincon:sqp}&6.37e+00&6.58e+00&6.02e+00\\\hline
		\textbf{fmincon:active-set}&1.82e+01&1.63e+01&1.57e+01\\\hline
		\textbf{OPTI:ipopt}&5.03e+00&4.45e+00&4.37e+00\\\hline
	\end{tabular}
	\caption{Min time till convergence in milliseconds}
	\label{tbl:min time till convergence}
\end{table}


\section{Simulations with state noise}
\label{appendix:benchmarks trailer with noise}

\begin{table}[H]
	\centering
	\begin{tabular}{|l|c|c|c|c|}
		\hline
		&\textbf{demo5}&\textbf{demo2}&\textbf{demo3}\\\hline
		\textbf{nmpc-codegen (PANOC)}&5.82e+00&4.20e-01&2.20e-01\\\hline
		\textbf{ForBeS zerofrp2 (PANOC)}&8.69e+01&1.19e+01&9.64e+00\\\hline
		\textbf{PANOC draft}&2.77e+01&4.54e+00&3.24e+00\\\hline
		\textbf{fmincon:interior-point}&3.17e+02&1.01e+02&8.15e+01\\\hline
		\textbf{fmincon:sqp}&9.07e+01&4.77e+01&3.93e+01\\\hline
		\textbf{fmincon:active-set}&2.29e+02&7.50e+01&6.46e+01\\\hline
		\textbf{OPTI:ipopt}&9.52e+01&1.41e+01&1.09e+01\\\hline
	\end{tabular}
	\caption{Mean time till convergence in milliseconds}
	\label{tbl:mean time till convergence with noise}
\end{table}

\begin{table}[H]
	\centering
	\begin{tabular}{|l|c|c|c|c|}
		\hline
		&\textbf{demo5}&\textbf{demo2}&\textbf{demo3}\\\hline
		\textbf{nmpc-codegen (PANOC)}&100&100&100\\\hline
		\textbf{ForBeS zerofrp2 (PANOC)}&1494&2839&4383\\\hline
		\textbf{PANOC draft}&476&1080&1472\\\hline
		\textbf{fmincon:interior-point}&5453&24054&37058\\\hline
		\textbf{fmincon:sqp}&1560&11360&17870\\\hline
		\textbf{fmincon:active-set}&3937&17860&29378\\\hline
		\textbf{OPTI:ipopt}&1637&3362&4952\\\hline
	\end{tabular}
	\caption{Mean relative time till convergence in milliseconds}
	\label{tbl:mean relative time till convergence with noise}
\end{table}

\begin{table}[H]
	\centering
	\begin{tabular}{|l|c|c|c|c|}
		\hline
		&\textbf{demo5}&\textbf{demo2}&\textbf{demo3}\\\hline
		\textbf{nmpc-codegen (PANOC)}&1.06e+02&1.30e+01&1.80e+01\\\hline
		\textbf{ForBeS zerofrp2 (PANOC)}&9.82e+02&2.23e+02&3.87e+01\\\hline
		\textbf{PANOC draft}&2.38e+02&7.07e+01&1.83e+01\\\hline
		\textbf{fmincon:interior-point}&2.45e+03&4.68e+02&3.40e+02\\\hline
		\textbf{fmincon:sqp}&4.22e+02&3.25e+02&4.18e+02\\\hline
		\textbf{fmincon:active-set}&1.13e+03&4.86e+02&5.69e+02\\\hline
		\textbf{OPTI:ipopt}&4.78e+02&9.21e+01&3.67e+01\\\hline
	\end{tabular}
	\caption{Max time till convergence in milliseconds}
	\label{tbl:max time till convergence with noise}
\end{table}

\begin{table}[H]
	\centering
	\begin{tabular}{|l|	c|c|c|c|}
		\hline
		&\textbf{demo5}&\textbf{demo2}&\textbf{demo3}\\\hline
		\textbf{nmpc-codegen (PANOC)}&0.00e+00&0.00e+00&0.00e+00\\\hline
		\textbf{ForBeS zerofrp2 (PANOC)}&4.62e+00&3.55e+00&3.18e+00\\\hline
		\textbf{PANOC draft}&1.80e+00&1.26e+00&1.22e+00\\\hline
		\textbf{fmincon:interior-point}&4.85e+01&1.33e+01&1.05e+01\\\hline
		\textbf{fmincon:sqp}&1.44e+01&8.63e+00&9.95e+00\\\hline
		\textbf{fmincon:active-set}&3.48e+01&2.04e+01&1.72e+01\\\hline
		\textbf{OPTI:ipopt}&1.68e+01&4.67e+00&5.38e+00\\\hline
	\end{tabular}
	\caption{Min time till convergence in milliseconds}
	\label{tbl:min time till convergence with noise}
\end{table}

\chapter{Poster}
\includepdf[pages={-},fitpaper,rotateoversize]{poster.pdf}