\documentclass[master=ewit]{kulemt}
\setup{title={Newton-type operator splitting methods for embedded model predictive control},
  author={Willem Melis},
  promotor={Prof.\,dr.\,ir.\ Panagiotis Patrinos},
  assessor={Prof.\,dr.\,ir.\ Johan Suykens \\ Prof.\,dr.\,ir.\ Daan Huybrechs},
  assistant={dr.\,Pantelis Sopasakis}}
% The following \setup may be removed entirely if no filing card is wanted
\setup{filingcard,
  translatedtitle=,
  udc=621.3,
  shortabstract={TODO: Add the same abstract here, containing no more than 500
    words. \LaTeX\ commands can be used here. Blank lines (or the command
    \texttt{\string\pa r}) are not allowed!
    \endgraf endgraf}}
% Uncomment the next line for generating the cover page
% \setup{coverpageonly}

\setup{filingcard, % Do not change anything in this line!
	%
	% Give the dutch translation of the title of your master's thesis below between the { and }:
	translatedtitle={Newton-type operator splitsen methoden voor embedded model predictive control},
	%
	% UDC number depends on your discipline. see http://www.udcsummary.info/php/index.php to find the number.
	udc=621.3,
	%
	% Add between { and } a short abstract.
	% Empty lines or the commando \par are not admitted.
	% Be careful with special TeX-signs #$%&^_~{}\ !!
	shortabstract={%
    The goal of this thesis is to implement a Matlab and Python library that generates a MPC controller in C that makes use of the PANOC algorithm. No toolboxes or external libraries were used to generated the C code, with the  exception of Casadi. Casadi is used to generate the gradient used in the proximal gradient algorithm. In the first chapters the PANOC algorithm is discussed. Followed by a high level view of nmpc-codegen, the software package constructed with this thesis. After the theory some simulation results are included that compare PANOC to some of the state of the art interior point and SQP solvers. PANOC dominates the simulations when the system equations are small, but loses this advantage when complex system equations are used. It does however always use less memory then the alternatives. The final chapter discuses a recent improvement of the PANOC algorithm. When adding soft constraints onto the problem, the weights of theses constraints must be set. Setting the weights too low will result in violations of those constraints, and setting it too worsens the condition of the problem. Recent research surrounding PANOC and the augmented Lagrangian sudgest that by defining and solving the MPC problem iteratively, the performance can be significantly improved. It is also much more intuitive to tune the augmented Lagrangian then to tune the weights of the constraints.	
}}

% Choose the main text font (e.g., Latin Modern)
\setup{font=lm}

\bibliographystyle{IEEEtran}

\usepackage{algorithm} 
\usepackage{algcompatible}
\usepackage{algpseudocode}
%\usepackage[parfill]{parskip} % paragraphs are seperated with an enter space

% Finally the hyperref package is used for pdf files.
% This can be commented out for printed versions.
\usepackage[pdfusetitle,colorlinks,plainpages=false]{hyperref}

\usepackage{graphicx}
\graphicspath{{./figs/}}
\usepackage{caption}
\usepackage{subcaption}
\usepackage{amssymb}
\usepackage{listings}
\usepackage{tabularx}

% package used to import poster
\usepackage[final]{pdfpages}

\usepackage{amsmath,amsthm,amssymb}
\DeclareMathOperator{\project}{project}
\DeclareMathOperator{\prox}{prox}
\DeclareMathOperator{\boxfunction}{box}
\DeclareMathOperator{\argmax}{argmax}
\DeclareMathOperator*{\minimize}{minimize}
\DeclareMathOperator*{\argmin}{arg\,min}
\DeclareMathOperator{\other}{other}

\newcommand{\Lagr}{\mathcal{L}} % this uses the amssym package

%\includeonly{chap-n}
\begin{document}

\begin{preface}
I would like to thank everyone that contributed in any way to this thesis, especially my promoter P. Patrinos and mentor P. Sopasakis. I would also like to thank the jury for reading the text. Finally i would like to thank my family for the moral support they gave me this last year.
\end{preface}

\tableofcontents*

\begin{abstract}
  The \texttt{abstract} environment contains a more extensive overview of
  the work. But it should be limited to one page.

\end{abstract}

% A list of figures and tables is optional
\listoffigures
\listoftables
% If you only have a few figures and tables you can use the following instead

% The list of symbols is also optional.
% This list must be created manually, e.g., as follows:
\listoffiguresandtables
\chapter{List of Abbreviations and Symbols}
\section*{Abbreviations}
\begin{flushleft}
  \renewcommand{\arraystretch}{1.1}
  \begin{tabularx}{\textwidth}{@{}p{12mm}X@{}}
    LoG   & Laplacian-of-Gaussian \\
    MSE   & Mean Square error \\
    PSNR  & Peak Signal-to-Noise ratio \\
  \end{tabularx}
\end{flushleft}
\section*{Symbols}
\begin{flushleft}
  \renewcommand{\arraystretch}{1.1}
  \begin{tabularx}{\textwidth}{@{}p{12mm}X@{}}
    42    & ``The Answer to the Ultimate Question of Life, the Universe,
            and Everything'' according to \cite{h2g2} \\
    $c$   & Speed of light \\
    $E$   & Energy \\
    $m$   & Mass \\
    $\pi$ & The number pi \\
  \end{tabularx}
\end{flushleft}

% Now comes the main text
\mainmatter

%\include{./extra/intro}
\chapter{Control Systems}
\section{Optimal control}
	The optimal control problem goes back more than 300 years. From Galileo and Newton to Johann Bernoulli and Euler with the brachistochrone problem. The optimal control problem can be simplified to finding the proper inputs so that the system behaves in a certain way, or put differently, find the control law.
	
	A typical optimal control problem exists out of a set of differential equations that describes the behavior of the system. And a cost function that describes the cost of the specific trajectory integrated on the differential equations. The most optimal solution to the optimal control problem will have the lowest cost possible.
	
	Optimal control problems can be mathematically defined as the cost function equation~\ref{eq:optimal control definition}(With boundary conditions equation~\ref{eq:optimal control definition state equations boundary conditions})  and the behavior of the system equation~\ref{eq:optimal control definition state equations}. In addition constraints can be placed on the state and inputs, as shown by equation~\ref{eq:optimal control definition state equations path constraints}. 
	
	\begin{equation}
		J = S[x(t_0),t_0,x(T),T] + \int_{t_0}^{T} L[x(t),u(t),t]
		\label{eq:optimal control definition}
	\end{equation}
	\begin{equation}
		\dot{x}(t) = F(x(t),u(t),t)
		\label{eq:optimal control definition state equations}
	\end{equation}
	\begin{equation}
		C[x(t),u(t),t]\le 0
		\label{eq:optimal control definition state equations path constraints}
	\end{equation}
	\begin{equation}
		S[x(t_0),t_0,x(T),T]=0
		\label{eq:optimal control definition state equations boundary conditions}
	\end{equation}

\section{MPC}
	Model predictive control is an advanced process control method that continuously solves a  optimal control problem. At a constant rate the state is measured, a optimal control problem is define and solved. The  optimal input is applied to the system, and the cycle repeats itself.
	
	A continuous physical system can be defined as $\dot{x}=F_c(x,u)$, where x is the current state and u is the current input. However when using computer systems its often required to have a discrete system. By using a discrete integrator the system can be written as $x^{k+1}=F_d(x^{k},u^{k})$. 
	
	Figure~\ref{fig:MPC diagram} is a diagram from \cite{Wikipedia} that illustrates MPC. The reference trajectory is the behavior we want, the predicted control input is the control input that was calculated and in the simulations lead to the predicted output.
	\begin{figure}[h]
		\centering
		\includegraphics[width=0.5\textwidth]{MPC_scheme}
		\caption{A simple MPC diagram from the wikipedia page \cite{Wikipedia}}
		\label{fig:MPC diagram}
	\end{figure}
			
	\subsection{System}
		
	\subsection{Problem definition}
	The problem definition is bases on the definition from \cite{Diehl2005}.
		\subsubsection{Problem form}
			The goal is to define the problem as equation~\ref{eq:PANOC MPC form}, and then solve it for u given $x_0$, the current state of the system. Sometimes $x_0$ will be assumed to be part of the function f, just like the reference state and input. Which leads to the simplified equation, equation~\ref{eq:PANOC form} .
			\begin{equation}
				\underset{u}{\minimize} \  f(x_0,u) + g(u)
				\label{eq:PANOC MPC form}
			\end{equation}
			
			\begin{equation}
				\underset{u}{\minimize} \  f(u) + g(u)
				\label{eq:PANOC form}
			\end{equation}
		\subsubsection{Direct Single shoot}
			If the horizon is N then the problem is solved for the inputs $u=[u_0,u_1,... u_{N-1}]$ where each $u_k$ is a vector of all the inputs of the system. This means that the vector u is of size Horizon $\cdot$ dimension\_input .
			
			The cost for each step in the horizon is defined as \ref{eq:single shot iteration cost}, this is called the stage cost.
			\begin{equation}
				\begin{aligned}
				& l_k(x_0,u) = &&  x_k^T Q x_k  +  u_k^T R u_k \\
				& \text{subject to}			&& x_0 = \bar{x} \\
				& 							&&  x_{n+1} = F_d(x_n,u_n), n=0...N-1
				\end{aligned}
				\label{eq:single shot iteration cost}
			\end{equation}
			
			The terminal cost is a special case of the stage cost as it is the last stage cost in the horizon. So if the Horizon=N the terminal cost can defined as equation~\ref{eq:single shot terminal cost}.
			
			\begin{equation}
				\begin{aligned}
					& l_N(x_0,u) = && x_N^TSx_N \\
					& \text{subject to}			&& x_0 = \bar{x} \\
					& 							&&  x_{n+1} = F_d(x_n,u_n), n=0...N-1
				\end{aligned}
				\label{eq:single shot terminal cost}
			\end{equation}
			
			$f(x_0,u)$ can then be defined as in equation~\ref{eq:single shot definition}, the sum of the stage costs and the terminal cost.
			\begin{equation}
				f(x_0,u) = \sum_{k=1}^{N-1} l_k(x_0,u) + l_N
				\label{eq:single shot definition}
			\end{equation}
			
			As a side note, an other term can be added to equation~\ref{eq:single shot definition} to represent the obstacle avoidance. More on this later on in this chapter in the subsection on obstacles.
		\subsection{Direct Multiple shoot}
			The multiple shoot needs more information than just an initial state. It requires an initial estimate of all of the intermediate states. The state estimates will be referred to as $x_i$ and the states derived from the estimate and its corresponding input will be referred to as $\bar{x_i}$. As before the goal of the optimization algorithm is to find the optimal inputs $u_i$, so that the cost function is as low as possible. And additionally  $\bar{x_i} - x_{i+1} = 0$, called the continuity conditions.
			
			\begin{equation}
				\bar{x_i} = F(x_i,u_i)
				\label{eq:}
			\end{equation}
			
			The continuity conditions are displayed in equation~\ref{eq:continuety condition multiple shot} and were not necessary in single shot as there are no state estimates like $\bar{x}$. If when starting the algorithm the initial state estimates are relatively close to the solution, a significant speed increase can be accomplished. As information was incorporate into the algorithm.
			
			\begin{equation}
				\bar{x_i} - x_{i+1} = 0
				\label{eq:continuety condition multiple shot}
			\end{equation}
			
			The direct multiple shoot definition looks like equation~\ref{eq:multiple shot cost} and has an extra equality condition compared to the single shoot. This equality condition will be added as a soft constraints to the cost function. This is displayed in equation~\ref{eq:multiple shot cost with soft constraint} and will be used in the practical implementation.
			
			\begin{equation}
				\begin{aligned}
				L =  & \sum_{i=1}^{N} l(\bar{x_i},u_i) \\
				& \text{subject to}			&& \bar{x_i} = F(x_i,u_i) \\
				& 							&& \bar{x_i} - x_{i+1} = 0
				\end{aligned}
				\label{eq:multiple shot cost}
			\end{equation}
			
			\begin{equation}
			\begin{aligned}
			L =  & \sum_{i=1}^{N} l(\bar{x_i},u) + ||\bar{x_i} - x_{i+1}||\\
			& \text{subject to}			&& \bar{x_i} = F(x_i,u_i) \\
			\end{aligned}
			\label{eq:multiple shot cost with soft constraint}
			\end{equation}
			
		\subsection{Obstacle avoidance}
			The obstacle avoidance is based on the soft constraint definition described in \cite{AjaySathya2017}. It can be described as a set or an constraint.
			\subsubsection{As set}
				An obstacle can be defined as an open set, as illustrated by equation~\ref{eq:obstacle as open set}. It is defined by the intersection of a set of nonlinear inequalities.
				\begin{equation}
					O = {z \in \Re^d : h^i(z)>0,\ i \in N}
					\label{eq:obstacle as open set}
				\end{equation}
				
			\subsubsection{As constraint}
				\begin{equation}
					[z]_+ =  \max\{0,z\}
				\end{equation}
				
				The statement $h(x)<0$ is equivalent to saying $[h(x)]_+=0$, so equation~\ref{eq:obstacle as open set} is equivalent to setting equation~\ref{eq:obstacle as equality} to zero.
				
				\begin{equation}
					\Phi_0(z) =  \frac{1}{2} \prod_{i=1}^m \Big( [h^i(z)]_+ \Big)^2
					\label{eq:obstacle as equality}
				\end{equation}
				
				The gradient of equation~\ref{eq:obstacle as equality} is define as equation~\ref{eq:obstacle as equality}
				
				\begin{equation}
					\nabla \Phi =
					\begin{cases}
						\sum_{i=1}^{m} h^i(z)\prod_{j \ne i} \Big( [h^i(z)]_+ \Big)^2 \nabla h^i(z)
						& x \in O \\
						0 & else
					\end{cases}
					\label{eq:derivative obstacle as equality}
				\end{equation}
			
			\subsubsection{Polyhedral obstacle}
				A simple obstacle example of such an obstacle is a polyhedral as defined in equation~\ref{eq:polyhedral constraint}.
				\begin{equation}
					\prod \Big([b_i - a_i^t z]_+ \Big)^2 = 0
					\label{eq:polyhedral constraint}
				\end{equation}
			
			\subsubsection{Obstacle as soft constraint}
				The obstacle avoidance is added as a soft constraint to the cost function. As demonstrated in equation~\ref{eq:derivative obstacle as equality}, this definition is two times differentiable and so this does not break the condition that the cost functions needs a gradient to be solved with the PANOC algorithm.
			
		\subsection{Input constraints}
			An other important aspect of a MPC problem are input constraints. In practice inputs have to comply with the physical properties of the devices. Absurdly high or low input values might in theory lead to a fast solution, but are not feasible in practice.
			
			A major advantage of the PANOC algorithm is that it can take non linear or non convex constraints. As longs as the proximal operation is analytically defined on the constraint it is feasible. 
			
			A simple example is the indicator box function, which allows to set a maximum and minimum value on the inputs. (The indicator box function is defined in the appendix) The indicator box function as input constraint can demand that every feasible solution lies within the bounds of the user defined box.
\chapter{Proximal gradient method}
	\section{Proximal mapping}
		The proximal operator is defined as $\prox_{\gamma g}(x)= \underset{u}{\argmin}(g(u) + \frac{1}{2 \gamma}||u-x||^2_2)$ in \cite{NealParikh}. 
		
		\begin{itemize}
			\item if $h(x)=0$ then $\prox_{\gamma h}(x)=x$ 
			\item if $h(x)=I_c(x)$ where $I_c$ is define in \eqref{eq:indicator function}, the proximity mapping of the indicator function is the orthogonal projection on the set.
		\end{itemize}
		
		The indicator function: is defined in equation ~\eqref{eq:indicator function}.
		\begin{equation}
			I_c = 
			\begin{cases}
			0 & x \in C  \\
			\infty & x \notin C
			\end{cases}
			\label{eq:indicator function}
		\end{equation}
		
		The proximal mapping can be seen as a generalization of a projection. The Appendix contains an other important example, the indicator box function.
	
	\section{Gradient projected method}
		
		\begin{equation}
			\begin{aligned}
			& \underset{x}{\text{argmin}}
			& & f_0(x) \\
			& \text{subject to}
			& & g(x)=0
			\end{aligned}
			\label{eq:prox grad opti problem}
		\end{equation}
		
		The classical gradient decent method cannot be used to solve the problem of \eqref{eq:prox grad opti problem}. As this problem has a condition that must be met by the algorithms solution. If in each iteration the solution is projected on the space spanned by $g(x)=0$ the iteration of \eqref{eq:grad descent} becomes \eqref{eq:projected grad descent}. This algorithm is called the gradient projection method.
		
		\begin{equation}
			x^k = x^{k-1} - \gamma \nabla f(x^{k-1})
			\label{eq:grad descent}
		\end{equation}
		
		\begin{equation}
			x^k = \project_{g(x)=0}[ x^{k-1} - \gamma \nabla f(x^{k-1})]
			\label{eq:projected grad descent}
		\end{equation}
		
		If $g(x)$ is the indicator function of the set onto which is projected. Then  \eqref{eq:projected grad descent} can be written as \eqref{eq:proximal grad descent}, known as the proximal gradient method.
		
		\begin{equation}
				x^k = \prox_{\gamma g}[ x^{k-1} - \gamma \nabla f(x^{k-1})]
			\label{eq:proximal grad descent}
		\end{equation}
	
	\section{The proximal gradient method}
		\eqref{eq:prox grad problem} Can be solved with the proximal gradient method, sometimes called forward backward splitting (FBS) . If the proximal mapping of $g(x)$ is analytically defined. 
			\begin{equation}
			\underset{x}{\argmin} = f(x) + g(x)
			\label{eq:prox grad problem}
			\end{equation}
		
		Inspired by the projected gradient method, the proximal gradient method is defined as \eqref{eq:prox grad method} in \cite{NealParikh}. The $\gamma$ variable is the step size, in order to have a convergence of O(1/k) $\gamma \in(0,1/L)$. If $\gamma \in (1/L,2/L)$ the algorithm will still converge but then its no longer a majorization-minimization method. (more on this see \cite{NealParikh})
		
		\begin{equation}
			x^k = \prox_{\gamma g}\big( x^{(k-1)}- \gamma \nabla f(x^{(k-1)})\big)
			\label{eq:prox grad method}
		\end{equation}	
	
	\section{Proximal minimization algorithm}
		 \cite{QianYang} Contains a short proof that illustrates the proximal mapping as a fixed point minimization algorithm. A property of the conjugated function is used to derive the gradient, see appendix for the theorem.
		 \begin{proof}
		 	The iteration $x^{k+1}=\prox_g(x^k)$ will minimize the smoothed version of function f(x), a shorter version of this proof van be found in \cite{QianYang}. 
		 	\begin{align*}
		 	f_{\mu}
		 	&= \underset{y}{\inf}\Big\{ |y| +\frac{1}{2 \mu}(x-y)^2 \Big\} \\
		 	&=   \frac{1}{2 \mu}||x||^2 + \frac{1}{2\mu} 
		 	\underset{y}{\inf}\Big\{
		 	2 \mu f(y) - 2x^T y + ||y||^2
		 	\Big\} \\
		 	&=  \frac{1}{2 \mu}||x||^2 + \frac{1}{\mu} 
		 	\underset{y}{\sup}\Big\{
		 	x^T y  - \mu f(y) - \frac{1}{2} ||y||^2 \Big \} \\
		 	&= \frac{1}{2\mu }||x||^2 \frac{1}{\mu } \Big( \mu f + \frac{1}{2}||\cdot||^2 \Big)^* (x) \\
		 	\nabla  f_{\mu} 
		 	&= \frac{x}{\mu} - \frac{1}{\mu} \underset{y}{\argmax} 
		 	\Big \{ x^Ty - \mu f(y) - \frac{1}{2}||y||^2 \Big \}\\
		 	& = \frac{1}{\mu}(x - \prox_{\mu f}(x)) \\
		 	\prox_{\mu f}(x)
		 	& = x- \mu \nabla f_{\mu}(x)
		 	\end{align*}
		 	\label{prf:proximal minimiztion alg proof}
		 \end{proof}

		 This means thats the iteration $x^{k+1}=\prox_g(x^k)$ will minimize the smoothed version of function g(x). 

\section{Proximal gradient method with line search}
	\subsection{Starting value gamma }
		\subsubsection{Estimating Lipschitz value}
			The Lipschitz of $\nabla f(x)$ value is a non negative number that complies with \eqref{eq:definition lipschitz value}.
			\begin{equation}
			L = \underset{x \neq y}{\sup} \frac{|\nabla f(y)-\nabla f(x)|}{|y-x|}
			\label{eq:definition lipschitz value}
			\end{equation}
			
			In practice it is not always possible to find the actual Lipschitz value. So the Lipschitz value is estimated ($L'$) locally at the starting location $x_0$ using \eqref{eq:estimated lipschitz value in starting position}. With $\delta=max[\delta_l,10^{-6} \cdot x_0]$ where $\delta_l$ is a small number chosen by the controller designer.
			% in code the safety value is DELTA_LIPSCHITZ_SAFETY_VALUE and set to 10^-6
			
			\begin{equation}
			L' = \frac{|\nabla f(x_0+\delta)-\nabla f(x_0)|}{|\delta|}
			\label{eq:estimated lipschitz value in starting position}
			\end{equation}
			
			The Lipschitz value is not explicitly saved but is used to estimate the line-search parameter $\gamma$. The backtracking of the proximal gradient algorithm then further improves $\gamma$. As the algorithm progresses and $\gamma$ improves, so does the estimation of the Lipschitz value indirectly.
		
		\subsubsection{Estimating gamma}	
			\cite{LorenzoStella2017} states that $\gamma<\frac{1}{L}$ guarantees convergence to a local minimum. As gamma needs to be smaller than $\frac{1}{L}$ a safety value is introduced. This idea was copied over from the kul-forbes/ForBES library by Lorenzo Stella and Panos Patrinos, which uses a $\beta$ of 0.05. And leads to \eqref{eq:starting value gamma}.
			\begin{equation}
			\gamma = \frac{1-\beta}{L}
			\label{eq:starting value gamma}
			\end{equation}		
	
	\subsection{Backtracking gamma}			
		\subsubsection{Backtracking in proximal gradient descent used in FBS}
			The line-search makes use of the quadratic bound of \eqref{eq:Armijo's sufficient decrease condition prox grad PANOC} Which is an quadratic bound, by backtracking on $\gamma$ the optimal step size is retrieved.
			
			\begin{equation}
			f({\bar{x}}) \leq f(x) - \nabla f(x)^T[x-\bar{x}] + \frac{1}{2 \gamma}||x-\bar{x}||^2
			\label{eq:Armijo's sufficient decrease condition prox grad PANOC}
			\end{equation}
			
			This leads to algorithm~\ref{alg:backtracking on gamma} the proximal gradient method. \eqref{eq:Armijo's sufficient decrease condition prox grad PANOC} can be seen as a quadratic model, as the Lipschitz value of the gradient is equal to $L=\frac{1}{\gamma}$.
			
			\begin{algorithm}
				\caption{backtracking $\gamma$}
				\label{alg:backtracking on gamma}
				\begin{algorithmic}[1]
					\Procedure {lineasearch\_gamma}{x,$\gamma$}
					\State $\bar{x} = \prox_{\gamma g}\big( x- \gamma \nabla f(x)\big)$
					\While{$f({\bar{x}}) > f(x) - \nabla f(x)^T[x-\bar{x}] + \frac{1}{2 \gamma}||x-\bar{x}||^2$}
					\State $\gamma = \frac{\gamma}{2}$
					\State $\bar{x} = \prox_{\gamma g}\big( x- \gamma \nabla f(x)\big)$
					\EndWhile
					\State \Return $\gamma$
					\EndProcedure
				\end{algorithmic}
			\end{algorithm}
	\subsection{Final algorthm}
		The final algorithm~\ref{alg:proximal gradient PANOC with backtracking} delivers the upward direction. $x_{k+1}=x_k - direction$.
		\begin{algorithm}
			\caption{proximal gradient PANOC with backtracking}
			\label{alg:proximal gradient PANOC with backtracking}
			\begin{algorithmic}[1]
				\Procedure{get\_proximal\_gradient\_step}{x,$\gamma$}
				\State $\gamma$=LINESEARCH\_GAMMA($\gamma$)
				\State $\bar{x} = \prox_{\gamma g}\big( x- \gamma \nabla f(x)\big)$
				\State \Return direction=$[x-\bar{x}]$, $\gamma$
				\EndProcedure
			\end{algorithmic}
		\end{algorithm}
\section{Proximal gradient alternative view}
	\subsection{Majorization-minimization algorithm}
	A Majorization-minimization algorithm is a type of algorithm that minimizes a surrogate function. The surrogate function is a approximation of the actual problem. And needs to have the conditions described in \eqref{eq:MM algorithm conditions}, where h(x,y) is a surrogate function and f(x) is the actual problem. The problem described in \eqref{eq:MM algorithm formula step} is iteratively solved until convergence.
	\begin{equation}
		\begin{cases}
			h(x,x) = f(x) \\
			h(x,y) \leq f(x)
		\end{cases}
		\label{eq:MM algorithm conditions}
	\end{equation}
	
	\begin{equation}
		x_{k+1} = \argmin_y h(x_k,y)
		\label{eq:MM algorithm formula step}
	\end{equation}

	\subsection{The proximal gradient as majorization-minimization algorithm}
	\cite{NealParikh} States that the proximal gradient algorithm can be seen as a majorization-minimization algorithm. Its surrogate function is described in \eqref{eq:surrogate function}, proof~\ref{prf:proximal gradient as MM} illustrates how minimizing the surrogates function is equal as taking a step of the proximal gradient algorithm.
	
	\begin{equation}
		h(x,y) = f(x) + \nabla f(x)^T(x-y) + \frac{1}{2 \cdot \gamma}||x-y||^2_2
		\label{eq:surrogate function}
	\end{equation}
	
	\begin{proof}
		$h(x,y) = f(x) + \nabla f(x)^T(x-y) + \frac{1}{2 \cdot \gamma}||x-y||^2_2$ is the MM surugate function of proximal gradient algorithm. The gradient is set too zero and solved for y.
		\begin{align*}
		h(x,y)
		& = f(x) + \nabla f(x)^T(x-y) + \frac{1}{2 \cdot \gamma}||x-y||^2_2 \\
		\frac{\partial h(x,y)}{\partial y}
		& = 0 + \nabla f(x) + \frac{1}{2 \cdot \gamma} + \frac{1}{2 \cdot \gamma}\frac{\partial ||x-y||^2_2}{\partial y}  \\	
		\frac{\partial ||x-y||^2_2}{\partial y}
		& = \frac{\partial(x-y)^T(x-y)}{\partial y} = \frac{\partial(-2x^Ty + y^Ty)}{\partial y} = -2x+y \\
		\frac{\partial h(x,y)}{\partial y}
		& = \nabla f(x) - \frac{x}{\gamma} + \frac{y}{2 \gamma} = 0 \\
		0
		& = \gamma \nabla f(x) - x + \frac{y}{2} \\
		y & = \gamma \nabla f(x) - x
		\end{align*}
		\label{prf:proximal gradient as MM}
	\end{proof}
\chapter{PANOC algorithm}
	This section is based on \cite{LorenzoStella2017} and \cite{AjaySathya2017}, the difference is that this text is focused implementation. And so the formula's often look slightly different. The Forward backward envelop part is based on \cite{Themelis}, which contains a slightly different algorithm that also uses the Forward backward envelop.
	\section{Introducing PANOC}
		The PANOC algorithm is an accelerated version of the proximal gradient descent algorithm. The direction  $x-\bar{x}$ is a convex combination of the proximal gradient algorithm's direction and an accelerator that makes use of curvature information. It is formally defined in \cite{LorenzoStella2017} as \eqref{eq:weighting linesearch FBE}, with $\tau_k \leq 1$.
		
		\begin{equation}
			x_{k+1} = x_k + (1-\tau_k)\cdot (x-\bar{x}) + \tau_k d_k
			\label{eq:weighting linesearch FBE}
		\end{equation}
		
		The term $\tau_kd_k$ can accelerate the convergence if $\tau_k\neq0$. The step $d_k$ is calculated using a quasi-newton algorithm. As a quasi-newton method uses curvature information of the cost function, it uses information not available to a gradient descent based method. 
		
		Furthermore it has a super linear convergence rate when it gets close to the solution. Which is much faster then the sub-linear convergence of the gradient descent algorithm . On top of that, the quasi-newton method does not require additional function or gradient evaluations. Which are the major contributors to the computational costs of the algorithm.
		
	\section{Quasi newton method}
		\subsubsection{Problem definition}
			The residue of the iteration \eqref{eq:prox grad method} can be used to solve the optimization problem.  By setting the residue in \eqref{eq:residue prox grad method} to zero, a local optimum can be obtained.
			
			\begin{equation}
			R_{\gamma}(x)= \frac{1}{\gamma}\left[ x - \prox_g( x - \nabla f(x)\gamma) \right]
			\label{eq:residue prox grad method}
			\end{equation}
			
			The root of \eqref{eq:residue prox grad method} can be obtained using the Newton iteration of \eqref{eq:newton iteration residual}. Where $H_k$ satisfies the inverse secant condition of \eqref{eq:newton iteration residual inverse secant}. If the implementation is aimed at embedded software L-BFGS is an excellent choice to solve \eqref{eq:residue prox grad method}. As it doesn't explicitly save the Hessian matrix.
			
			\begin{equation}
			x^{k+1} = x^k -H_kR_{\gamma}(x^k)
			\label{eq:newton iteration residual}
			\end{equation}
			\begin{equation}
			x^{k+1} - x^k = H_{K+1} \Big( R_{\gamma}(x^{k+1})- R_{\gamma}(x^k) \Big)
			\label{eq:newton iteration residual inverse secant}
			\end{equation}
		
	\section{Forward backward envelop}	
		Newton iterations only converge when they are close enough to the solution. In order to get good global behavior, a proper global strategy is required. The optimization problem is changed from $\varphi(x) = f(x) + g(x)$ to \eqref{eq:FBE definition using Moreau envelope}. This problem is smoother while it still has the same optimal solution.(proof see \cite{LorenzoStella2017} and \cite{Themelis}) The same $\gamma$ as with the proximal gradient should be used, notice how the FBE contains the line-search condition used in the proximal gradient. (more on the FBE in \cite{Themelis})
		
		The Moreau envelope is de defined as \eqref{eq:Moreau envelope}, this smooths a function. Using simple algebra \eqref{eq:FBE definition using Moreau envelope} can be transformed into \eqref{eq:FBE definition using quadratic approximation}. The solution $y$ of the infimum in \eqref{eq:FBE definition using quadratic approximation} is $\bar{x}$. Considering the close relationship between the Moreau envelope and the proximal operator this is to be expected. (see more in \cite{Themelis}) 
		
		An alternative way to look at the FBE is illustrated in \cite{AjaySathya2017}, where the problem can be seen as minimizing a quadratic approximation  $f(x) +  \nabla f(x)^T(y-x) + g(y) + \frac{1}{2 \gamma} ||x-y||^2  $ towards y in point x.  Because $L = \frac{1}{\gamma}$, with L as the Lipschitz constant of the gradient.
		
		\begin{equation}
			g^{\gamma} = \underset{y}{\inf} \big \{f(y)+\frac{1}{2 \cdot \gamma}||x-y||^2 \big \}
			\label{eq:Moreau envelope}
		\end{equation}
		
		\begin{equation}
		\varphi_{\gamma} = f(x) - \frac{\gamma}{2}||\nabla f(x)||^2 + g^{\gamma} \big(x-\gamma \nabla f(x) \big)
		\label{eq:FBE definition using Moreau envelope}
		\end{equation}
		
		\begin{equation}
		\varphi_{\gamma} =   f(x) + \underset{y}{\inf} \Big\{ \nabla f(x)^T(y-x) + g(y) + \frac{1}{2 \gamma} ||x-y||^2  \Big\}
		\label{eq:FBE definition using quadratic approximation}
		\end{equation}
		
		\begin{proof}
			The solution to the infimum of \eqref{eq:FBE definition using quadratic approximation} is y=$\bar{x}=\prox_g(x-\gamma \nabla f(x))$
			\begin{align*}
			\prox_g(\bar{x}) 
			&=\prox_g(x- \gamma \nabla f(x)) \\
			&= \underset{y}{\argmin} \Big \{ g(y) 
			+ \frac{1}{2 \gamma}||(y-x) + \gamma \nabla f(x)||^2 \Big \} \\
			&= \underset{y}{\argmin} \Big \{ g(y) 
			+ \frac{1}{2 \gamma} \big[||y-x||^2 + 2 \gamma \nabla f(x)^T(y-x) + ||\nabla f(x)||^2\gamma^2 \big] \Big \} \\
			&= \underset{y}{\argmin} \Big \{ g(y) 
			+ \frac{1}{2 \gamma} \big[||y-x||^2 + 2 \gamma \nabla f(x)^T(y-x)  \big] \Big \}\\
			&= \underset{y}{\argmin} \Big \{   \nabla f(x)^T(y-x)  + g(y) 
			+ \frac{1}{2 \gamma} ||y-x||^2  \Big  \}
			\end{align*}
			\label{prf:prox is solution to FBE inf}
		\end{proof}
		
		\eqref{eq:practical implementation of FBE} Is the practical implementation of the FBE. The parameter gamma is the line-search parameter used in the proximal gradient descent. The first 3 terms are the same as with the line-search on $\gamma$. The last term $g(\bar{x})$ is new and ensures that the solution complies with the constraint.
		
%		\cite{AjaySathya2017} has an excellent example to illustrate this.(Figure~\ref{fig:FBE illustration}).
%		
		\begin{equation}
			\begin{aligned}	
				& \varphi(\gamma,x)= 
				&& f(x) - \nabla f(x)^T(x-\bar{x}) + \frac{1}{2 \gamma}||x-\bar{x}||^2  + g(\bar{x})
				\\
				& with 
				&&\bar{x} = \prox_g( x - \gamma\nabla f(x)) 
			\end{aligned} 
			\label{eq:practical implementation of FBE}
		\end{equation}
		
%		\begin{figure}[H]
%			\centering
%			\label{fig:FBE illustration}
%			\includegraphics[width=0.6\textwidth]{FBE}
%			\caption{FBE example copied over from \cite{AjaySathya2017}, red is?? blue is ?? TODO expain this figure}
%		\end{figure}
	
	\section{Line-search with FBE}
	In \cite{LorenzoStella2017} the line-search condition is specified as equation ~\eqref{eq:line-search with FBE}. The line-search parameter $\tau$ determines $x^{k+1}$ by weighting the convex combination of the proximal gradient step and the L-BGFS step as defined in \eqref{eq:linea-search tau definition}.  \cite{LorenzoStella2017} Specifies that $\sigma \in (0, \gamma \frac{1-\gamma\cdot L}{2})$. As stated before it is assumed that $L=\frac{1-\beta}{\gamma}$, some simple algebra will lead to the condition $\sigma \in (0,\frac{\beta \gamma}{2})$.
	
	 \eqref{eq:practical line-search with FBE} Is the practical implementation of \eqref{eq:line-search with FBE}, it incorporates the safety value $\beta$. A new constant $\alpha \in (0,1)$ is introduced, a possible value for $\alpha$ would be 0.5 ($\alpha=0.5$ is the choice used by the Matlab implementation of PANOC in ForBes known as zerofpr2).
	
	\begin{equation}
		x^{k+1} = u_k - (1-\tau_k)\cdot (x-\bar{x}) + \tau \cdot dir_{LBFGS}
		\label{eq:linea-search tau definition}
	\end{equation}
	
	\begin{eqnarray}
		\label{eq:line-search with FBE}
		\varphi_{\gamma}(x^{k+1})\leq\varphi_{\gamma}(x^{k}) - \sigma ||\frac{x-\bar{x}}{\gamma}||^2 \\
		=
		\varphi_{\gamma}(x^{k}) - \frac{\sigma}{\gamma^2} ||x-\bar{x}||^2
	\end{eqnarray}
	
	\begin{equation}
		\varphi_{\gamma}(x^{k+1}) \leq 		\varphi_{\gamma}(x^{k}) - \alpha \frac{\beta }{\gamma \cdot 2} ||x-\bar{x}||^2
		\label{eq:practical line-search with FBE}
	\end{equation}
	
	
		
		\begin{algorithm}
			\caption{PANOC}
			\label{alg:PANOC}
			\begin{algorithmic}[1]
				\Procedure {PANOC\_GET\_NEW\_LOCATION}{$x^k$,$\gamma$}
				\State [$(x-\bar{x})$ , $\gamma$] = GET\_PROXIMAL\_GRADIENT\_STEP($\gamma$,$x^k$)
				\State $ dir_{LBFGS}$ = LBFGS($x^k$)
				\State $\tau =1$
				\State $x^{k+1} = x_k - (1-\tau_k)\cdot (x-\bar{x}) + \tau \cdot dir_{LBFGS}$
				\While{$\varphi_{\gamma}(x^{k+1}) > 		\varphi_{\gamma}(x^{k}) - \alpha \frac{\beta}{\gamma \cdot 2} ||(x-\bar{x})||^2$}
				\State $\tau = \tau / 2$
				\State $x^{k+1} = x_k - (1-\tau_k)\cdot (x-\bar{x}) + \tau \cdot dir_{LBFGS}$
				\EndWhile
				\EndProcedure
			\end{algorithmic}
		\end{algorithm}
	
\chapter{L-BFGS}
	In this chapter the implementation of solving \eqref{eq:LBFGS problem} with L-BFGS is discussed. A naive implementation can lead to strange behavior. So it's important to spend time on the L-BFGS implementation.
	
	
	\begin{equation}
	R(x) = \frac{1}{\gamma}\left[ x - \prox_g( x - \nabla f(x)\gamma) \right] = 0
	\label{eq:LBFGS problem}
	\end{equation}
	
	\section{Exact newton method}
	The exact Newton method can be derived from a Taylor expansion. It's quadratic convergence makes it a popular choice for solving optimization problems. However convergence is not guaranteed and only quadratic if it is close enough to the solution. The Hessian matrix must be available to the algorithm, and is used in each iteration of the algorithm to solve a system.
	
		\begin{equation}
			g(x_{k+1}) = g(x_k) + \nabla g(x_k)(x_{k+1}-x_k)
			\label{eq:Taylor expansion}
		\end{equation}
		
		\begin{equation}
			x_{k+1}-x_k = -(\nabla g(x_k))^{-1} \cdot g(x_k)
			\label{eq:Taylor expansion reshaped}
		\end{equation}
		
	The Taylor expansion in point $x^k$ can be written as \eqref{eq:Taylor expansion}, if $g(x^{k+1})$ is set to zero, the equation becomes \eqref{eq:Taylor expansion reshaped}. This leads to \eqref{eq:newton method} where $g(x)=\nabla f(x)$ and the step towards zero is p. Typically some sort of line search used to accelerate convergence, so the next location is found using \eqref{eq:newton method linesearch}, with linesearch parameter.	
	
		\begin{equation}
			p = x_{k+1}-x_k = -(\nabla^2 f(x_k))^{-1} \cdot \nabla f(x_k)
			\label{eq:newton method}	
		\end{equation}
	
		\begin{equation}
		 	x_{k+1} = x_k + \alpha p
		 	\label{eq:newton method linesearch}
		\end{equation}
		
	\section{Quasi Newton Method:BFGS}
	As mentioned in the previous section in order to use the exact Newton method the hessian matrix must be available in practice. Sometimes this matrix exists but is not analytically available. It might be hard to express analytically, or too large to save explicitly in memory. In either case it is possible to approximate the Hessian matrix as illustrated in \eqref{eq:quasi newton method approx Hessian}.  This leads to the secant condition \eqref{eq:secant condition}. One popular formulation that satisfies the secant condition is the BFGS algorithm.
		
		\begin{equation}
			B_{k+1}(x_{k+1}-x_k) = \nabla f(x_{k+1}) - \nabla f(x_k)
			\label{eq:quasi newton method approx Hessian}	
		\end{equation}
		
		\begin{eqnarray}
			s_k = x_{k+1} - x_{k} \\
			y_k = \nabla f(x_{k+1}) - \nabla f(x_{k}) \\
			\rho_k = \frac{1}{y_k^T \cdot s_k}
		\end{eqnarray}
	
		
		\begin{equation}
			B_{k+1} s_{k} = y_{k}
			\label{eq:secant condition}
		\end{equation}
		
		\begin{equation}
			B_{k+1} = B_{k} + \frac{y_k y_k^T}{ y_k^T s_k} - \frac{B_k s_k s_k^T B_k^T}{s_k^TB_ks_k}
			\label{eq:quasi newton method approx Hessian with past values}
		\end{equation}
	Calculating the inverse of the B matrix is an expensive operation. Solving a system of linear equations has a complexity of $\mathcal{O}(n)^3$ ). \eqref{eq:quasi newton method inverse hessian} Expresses the inverse of the Hessian in a function of the past values. In every iteration the L-BFGS step is calculated using \eqref{eq:quasi newton method}. The new location can now be used to update the inverse hessian using \eqref{eq:quasi newton method inverse hessian}.
		\begin{equation}
			V_k = I - \rho_ky_ks_k^T
		\end{equation}
	
		\begin{equation}
			H^{k+1} = V_k^TH_kV_k + \rho_ks_ks_k^T
			\label{eq:quasi newton method inverse hessian}	
		\end{equation}
		
		\begin{equation}
		p = x_{k+1}-x_k = -(B_k)^{-1} \cdot \nabla f(x_k) = -H_k\cdot \nabla f(x_k)
		\label{eq:quasi newton method}	
		\end{equation}
		
	\section{Quasi Newton Method:L-BGFS}
	One variation of the BFGS algorithm is the L-BFGS algorithm. The L-BFGS algorithm does not express the Hessian matrix explicitly. This way less memory is required to run the algorithm. Which is very useful if either the problem is large, or the memory is small. As nmpc-codegen is aimed at embedded devices the latter is very important.
	
	\eqref{eq:quasi newton method inverse hessian} can be written out recursively as illustrated in \eqref{eq:quasi newton method inverse hessian recursive}. \eqref{eq:quasi newton method inverse hessian recursive} Can also be expressed in two loops as illustrated in algorithm~\ref{alg:LBFGS}.
	
		\begin{eqnarray}	 
			\begin{aligned}
				H_k = 
				& (V^T_{k-1} ... V^T_{k-m})H^0_k(V^T_{k-m} ... V_{k-1}) \\
				& + \rho_{k-m} (V^T_{k-1} ... V^T_{k-m+1})s_{k-m}s_{k-m}^T(V^T_{k-m+1} ... V_{k-1}) \\
				& + \rho_{k-m+1} (V^T_{k-1} ... V^T_{k-m+2})s_{k-m+1}s_{k-m+1}^T(V^T_{k-m+2} ... V_{k-1}) \\
				& + ... \\
				& + \rho_{k-1}s_{k-1}s_{k-1}^T
			\end{aligned}
			\label{eq:quasi newton method inverse hessian recursive}
		\end{eqnarray}
		
		\begin{equation}
			H_k^0 = \frac{s_k^Ty_k}{y_k^Ty_k}
			\label{eq:quasi newton method initial Hessian}
		\end{equation}
		
	The initial Hessian $H^0_k$ can be estimated using the most recent s and y values, as illustrated in \eqref{eq:quasi newton method initial Hessian}. If the optimal point is a local minimum, it's advisable to check if the new initial Hessian is positive definite. The next location then becomes $x_{k+1} = x_{k}+ \alpha \cdot p$ with alpha being the line search parameter. Typically either a wolf-type or Armijo-type line search is used, however PANOC uses neither of these methods so they are not discussed.
	
	\section{Cautious update L-BGFS}
	As mentioned before when updating the initial hessian value, it's good practice to check if it's positive. If it is negative, there's no point in updating the Hessian value. However even with this safety mechanism, it is still possible to get bad updates. As in updates that lower the convergence rate of future L-BFGS steps.
	
	\begin{equation}
		\frac{y_k^Ts_k}{s_k^Ts_k} \ge \epsilon ||\nabla f(x_k)||^\alpha
		\label{eq:cautious update}
	\end{equation}
	
	\cite{Dong-HuiLi1999} Suggests a cautious update based on the size of the gradient and $\frac{s_k^Ts_k}{s_k^Ts_k}$ as illustrated in \eqref{eq:cautious update}. The two parameters epsilon and alpha are both positive constants. The cautious update of \eqref{eq:cautious update} will be relevant when the algorithm is near fixed point. And the step size is so small the function values hardly change. The information in $s_k$ and $y_k$ is then dominated noise and cannot be trusted.
	
	\subsection{L-BFGS implementation}
		When using  L-BFGS algorithm(algorithm~\ref{alg:LBFGS}) to solve \eqref{eq:LBFGS problem}. The direction is equal to the proximal gradient step if there is no initial hessian estimate yet. And current\_buffersize will increase every valid update from 1 to the limit specified by the controller designer.
		
		\begin{equation}
			R(x) = \frac{1}{\gamma}\left[ x - \prox_g( x - \nabla f(x)\gamma) \right] = 0
			\tag{\eqref{eq:LBFGS problem} revisited}
		\end{equation}
		
		\begin{algorithm}
			\caption{LBFGS}
			\label{alg:LBFGS}
			\begin{algorithmic}[1]
				\Procedure {LBFGS}{$x^k$,M=current\_buffersize}
				\State $q = R(x^k)$
				\For{i=M:1}
				\State $\alpha(i)=\rho(i) \cdot s(:,i)^Tq$
				\State $q = q - \alpha(i) \cdot y(:,i)$
				\EndFor
				\State $H_k^0 = y(:,M) \cdot s(:,M)^T \cdot  \frac{1}{y(:,M)^T \cdot y(:,M)}$
				\State $H^0_k \cdot R(x^k)$
				\For{i=1:M}
				\State $\beta(i) = \rho(i) \cdot y(:,i)^T \cdot z$
				\State $z = z + s(:,i)[\alpha(i)-\beta(i)]$
				\EndFor
				\For{i=1:M-1}
				\State $s(:,i+1)=s(:,i)$
				\State $y(:,i+1)=y(:,i)$
				\EndFor
				\State $$\begin{cases}
				s(:,1) = x_{k+1} - x_k \\
				y(:,1) = \nabla f(x_{k+1}) - \nabla f(x_k)\\
				\rho_k(1) = \frac{1}{y(:,1)^T \cdot s(:,1)} \\ 
				\end{cases}
				$$
				\State \Return direction=$-z=-H_k \cdot R(x^k)$
				\EndProcedure
			\end{algorithmic}
		\end{algorithm}
	
	\subsection{Notes on implementing L-BFGS with PANOC}
	Even though the papers on the PANOC algorithm sudgest that the problem solved with L-BFGS is \eqref{eq:LBFGS problem}. Using the unnormalized $R'(x) = \left[ x - \prox_g( x - \nabla f(x)\gamma) \right]$ with the L-BFGS will give better numerical results.
	
	\begin{equation}
	R(x) = \frac{1}{\gamma}\left[ x - \prox_g( x - \nabla f(x)\gamma) \right] = 0
	\tag{\eqref{eq:LBFGS problem} revisited}
	\end{equation}
	
	Every time the search parameter $\gamma$ changes the L-BFGS buffer must be flushed.
\chapter{nmpc-codegen}

\section{Overview library}
	\begin{figure}[H]
		\centering
		\includegraphics[width=0.5\textwidth]{nmpc_codegen_scheme}
		\caption{nmpc-codegen scheme}
		\label{fig:nmpc-codegen scheme}
	\end{figure}

\section{Panoc implementation}
	\begin{figure}[H]
		\centering
		\includegraphics[width=0.5\textwidth]{visio_software_arch}
		\caption{Software architecture}
		\label{fig:visio software arch}
	\end{figure}

\chapter{Simulation}
This chapter discusses some simulation results, using a simple mathematical model to create a controller. First the mathematical model is discussed, after that the simulation results are discussed and compared with the internal Matlab solvers.

\section{Model complexity}
The trailer model has a very low computational complexity, which makes it cheap to simulate with. The quad copter model, is a very complex model, that requires lots of computations for each simulation step. In this chapter it will become clear that PANOC is a very good with simple models, but a significantly slower with complex models compared to a interior point method.


\section{Trailer model}
The trailer model is illustrated in figure~\ref{fig:trailer model}, the left rectangle is the trailer and the right rectangle is the driver. The driver pulls the trailer forward. The driver is connect to the trailer via a single arm as illustrated in figure~\ref{fig:trailer model}. The control system can only change the speed in the Y and X axis. The goal of the control system is to get the trailer at a certain position and at a certain angle. Figure shows the trailer in a neutral position, so under an angle of 0 degrees.

\begin{figure}
	\centering
	\includegraphics[width=0.5\textwidth]{trailer}
	\caption{trailer}
	\label{fig:trailer model}
\end{figure}

The mathematical model of the trailer is represented in equation~\ref{eq:trailer model}, $u_x$ and $u_y$ are the inputs of the system and represent the speed in the Y and X direction. The angle is represented by $\theta$ , the distance between the driver and the trailer is represented by L. The position of the trailer is represented by $p_x$ and $p_y$.

\begin{equation}
	\begin{cases}
		\dot{p_x} = u_x + L sin \theta \cdot \dot{\theta} \\
		\dot{p_y} = u_y + L cos \theta \cdot \dot{\theta} \\
		\dot{\theta} = \frac{1}{L}(u_ycos \theta - u_x sin \theta)	
	\end{cases}
	\label{eq:trailer model}
\end{equation}

\section{A simple trailer example}
A simple way of measuring the performance of the library is to compared it to the alternatives. In the following simulation the nmpc-codegen library will be compared to, ForBes zerofrp2 the Matlab implementation of the panoc algorithm. And to the internal Matlab function fmincon using either of the tree algorithms interior point,SQP and active set.

\subsection{General performance}
Several NMPC problems can be constructed and solved with nmpc-codegen. Figure~\ref{fig:demos} illustrates the problems used to benchmark nmpc-codegen, they all use the simple trailer model but have different obstacles. The time till convergence is measured, the average time is displayed in table~\ref{tbl:mean time till convergence} in the appendix. While the maximum and minimum time is displayed in the appendix: table~\ref{tbl:min time till convergence} and table~\ref{tbl:max time till convergence}.

The convergence time can be expresses relatively to the convergence time of the nmpc-codegen time as illustrated by equation~\ref{eq:definition relative time}. These results are displayed in table~\ref{tbl:mean relative time till convergence}. From table~\ref{tbl:mean relative time till convergence} it's clear that nmpc-codegen dominates by a significant margin.

\begin{equation}
	t_{relative} = \frac{t_{algorithm}}{t_{nmpc-codegen}}
	\label{eq:definition relative time}
\end{equation}

\subsection{In-dept analysis of performance}
The in-dept simulation contains four circular obstacles, the trailer will move from the lower left corner to the upper right corner. Each of the algorithms will calculate the optimal path up to the tolerance of $10^{-3}$. The input for one step is then applied to the system equation, and the next state is obtained. The same nmpc problem is solved using the new state as current state  and the new input is applied again.

The state of the trailer will is represented by arrows, the starting point of the arrow is the position of the trailer. The angle of the arrow represents the positional angle of the trailer. The amplitude of the arrow has no meaning.

Figure~\ref{fig:solution path trailer example} contains the results of the simulation. The panoc algorithm calculated the lowest located path while all three of the fmincon algorithms opted for the upper located path. Both of these paths are valid solutions. It is immediately clear from figure~\ref{fig:timings trailer example} that the nmpc-codegen library is significantly faster. The other four algorithm's are about the same for the first 30 steps, after about 30 steps the ForBes zerofpr2 algorithm is faster then either of the tree fmincon algorithms.

The nmpc-codegen library uses a cautious lbfgs update which improves the performance of the lbfgs. This is clear from figure~\ref{fig:iterations trailer example} where not the time to convergence but iterations till convergence are illustrated. The nmpc-codegen clearly needs less iterations than theForbes zerofpr2. As the simulation progresses, less and less iterations are needed to calculate optimal solution. And the lbfgs doesn't influence the convergence as much, so the two algorithms have about the same amount iterations till convergence.
\begin{figure}[H]
	\centering
	\begin{subfigure}[b]{0.45\textwidth}
		\centering
		\includegraphics[width=1.2\textwidth]{compare_libs/path}
		\caption{path}
		\label{fig:solution path trailer example}
	\end{subfigure}
	
	\begin{subfigure}[b]{0.45\textwidth}
		\centering
		\includegraphics[width=1.2\textwidth]{compare_libs/iterations}
		\caption{iterations}
		\label{fig:iterations trailer example}
	\end{subfigure}
	\hfill
	\begin{subfigure}[b]{0.45\textwidth}
		\centering
		\includegraphics[width=1.2\textwidth]{compare_libs/timings}
		\caption{timings}
		\label{fig:timings trailer example}
	\end{subfigure}
	\caption{Schematic representation of the software}
\end{figure}

\section{Quadcopter model}
%TODO: reference source model !!!!!!!!!!!!!!!!!!
The invention of electronic control systems has proven to be especially useful when controlling quadcopters. Because of their complex behavior they are hard to control manually by a pilot. However quadcopters are popular among unmanned aerial vehicles but they are controlled by a fast digital/analog controller.

\subsection{Mathematical model}
The behavior of a quad copter is described by a set of nonlinear equations as illustrated in equation~\ref{eq:mathematical model quadcopter}.

\begin{table}[]
	\centering
	\caption{constants used in quadcopter model}
	\label{btl:quadcopter model constants}
	\begin{tabular}{|l|l|l|l|}
		\hline
		Parameter                               & Symbol   & Value             & Unit           \\ \hline
		Mass of the quadcopter                  & m        & 0.5               & kg             \\
		Radius of the quadcopter                & L        & 0.25              & m              \\
		Propeller lift coefficient              & k        & $3 \cdot 10^{-6}$ & $Ns^2$        \\
		Propeller drag coefficient              & b        & $1 \cdot 10^{-7}$ & N m $s^2$      \\
		Acceleration of gravity                 & g        & 9.81              & m/$s^2$        \\
		Air friction coefficient                & $k_d$    & 0.25              & kg/s           \\
		Quadcopter inertia about the $x^b$-axis & $I_{xx}$ & $5 \cdot 10^{-3}$ & kg $m^2$       \\
		Quadcopter inertia about the $y^b$-axis & $I_{yy}$ & $5 \cdot 10^{-3}$ & kg $m^2$       \\
		Quadcopter inertia about the $z^b$-axis & $I_{zz}$ & $1 \cdot 10^{-2}$ & kg $m^2$       \\ 
		Motor constant                          & $c_m$    & $1 \cdot 10^{4}$  & $v^{-2}s^{-2}$ \\
		\hline
	\end{tabular}
\end{table}

\begin{equation}
	\begin{aligned}
		\dot{x} &= v_x \\
		\dot{y} &= v_y \\
		\dot{z} &= v_z \\
		\dot{v_x} &= -\frac{k_d}{m}v_x + \frac{k \cdot c_m}{m}\Big(sin(\gamma)sin(\phi)+cos(\gamma)cos(\phi)sin(\theta)\Big)\Big(u_1 + u_2 + u_3 + u_4\Big) \\
		\dot{v_y} &= -\frac{k_d}{m}v_y + \frac{k \cdot c_m}{m}\Big(cos(\phi)sin(\gamma)sin(\theta)-cos(\gamma)sin(\phi)\Big)\Big(u_1 + u_2 + u_3 + u_4\Big) \\
		\dot{v_z} &= -\frac{k_d}{m}v_y -g + \frac{k \cdot c_m}{m}\Big(cos(\theta)cos(\phi)\Big)\Big(u_1 + u_2 + u_3 + u_4\Big) \\
		\dot{\phi} &= \omega_x + \omega_y\Big( sin(\phi)tan(\theta) \Big) + \omega_z \Big( cos(\phi) tan(\theta) \Big) \\
		\dot{\theta} &= \omega_y cos(\phi) - \omega_z sin(\phi) \\
		\dot{\gamma} &= \frac{sin(\phi)}{cos(\theta)}\omega_y + \frac{cos(\phi)}{cost(\theta)} \omega_z \\
		\dot{\omega_x} &= \frac{Lkc_m}{I_{xx}}\Big( u_1 - u_3 \Big) - \Big( \frac{I_{yy}-I_{zz}}{I_{xx}} \Big) \omega_y \omega_z\\
		\dot{\omega_y} &= \frac{Lkc_m}{I_{yy}}\Big( u_2 - u_4 \Big) - \Big( \frac{I_{zz}-I_{xx}}{I_{yy}} \Big) \omega_x \omega_z\\
		\dot{\omega_z} &= \frac{bc_m}{I_{zz}}\Big( u_1 - u_2 + u_3 - u_4 \Big) - \Big( \frac{I_{xx}-I_{yy}}{I_{zz}} \Big) \omega_x \omega_y\\
	\end{aligned}
	\label{eq:mathematical model quadcopter}
\end{equation}

\subsection{Simulation results}
Figure~\ref{fig:Simulation results with quadcopter} contains a simple simulation with the quad copter model, the path is displayed in figure~\ref{fig:solution path trailer quad}. The two sphere shaped obstacles, and the quad copter is moving from the star symbol to the circle. The time to convergence of each step of the simulation, is displayed in figure~\ref{fig:timings trailer quad}.

As mentioned at the start of this chapter, the quad copter model computational expensive model. This means that the computational cost of solving a system, is more in line with the computational cost of simulating one step of the system. 

A interior point method, must solve a system at each step of the algorithm. The interior method used from ipopt solves the system directly with a computational cost of about $\mathcal{O}(n^3)$ . While PANOC only has to cope additional cost of about $\mathcal{O}(Ln)$ with L as the buffer size.  because the L-BFGS algorithm only uses inner products and vector additions.

This means that the difference between PANOC and ipopt is rather small with the quad copter model. Which is clearly visible, in figure~\ref{fig:timings trailer quad}, where the ipopt line is very near the ipopt line.
\begin{figure}[H]
	\centering
	\begin{subfigure}[b]{0.45\textwidth}
		\centering
		\includegraphics[width=1.2\textwidth]{compare_libs/path_quad}
		\caption{path}
		\label{fig:solution path trailer quad}
	\end{subfigure}
	\hfill
	\begin{subfigure}[b]{0.45\textwidth}
		\centering
		\includegraphics[width=1.2\textwidth]{compare_libs/timings_quad}
		\caption{timings}
		\label{fig:timings trailer quad}
	\end{subfigure}
	\caption{Simulation results with quadcopter}
	\label{fig:Simulation results with quadcopter}
\end{figure}

\section{Influence of noise}
Until now all simulations were executed without any kind of noise. However in reality there is always some kind of state noise. So is essential to study how PANOC handles state noise.

The white noise has a maximum amplitude of 0.1 on the x state and the y state. It has a amplitude of 0.05 on the angle $\theta$. The double noise simply doubles the maximum amplitude of the noise. This way the influence of the size of the noise can be studied.

Figure~\ref{fig:Noise simulations with the trailer model} contains the simulation results using PANOC or ipopt as solver. Going from no state noise to a bit of state noise makes a significant difference when solving with PANOC. The interior point method from ipopt, thus take longer to converge when noise is added. However it is less proportional towards the noise, as the interior point method here uses a direct method to solve the system.

\begin{figure}[H]
	\centering
	\begin{subfigure}[b]{0.45\textwidth}
		\centering
		\includegraphics[width=1.2\textwidth]{compare_libs/trailer_without_noise}
		\caption{without noise}
		\label{fig:timings trailer without noise}
	\end{subfigure}
	\hfill
	\begin{subfigure}[b]{0.45\textwidth}
		\centering
		\includegraphics[width=1.2\textwidth]{compare_libs/trailer_with_noise}
		\caption{with noise}
		\label{fig:timings trailer with noise}
	\end{subfigure}
	\begin{subfigure}[b]{0.45\textwidth}
		\centering
		\includegraphics[width=1.2\textwidth]{compare_libs/trailer_with_double_noise}
		\caption{with double noise}
		\label{fig:timings trailer with double noise}
	\end{subfigure}
	\hfill
	\begin{subfigure}[b]{0.45\textwidth}
		\centering
		\includegraphics[width=1.2\textwidth]{compare_libs/trailer_with_double_noise}
		\caption{path without noise}
		\label{fig:path noise simulations}
	\end{subfigure}
	\caption{Noise simulations with the trailer model}
	\label{fig:Noise simulations with the trailer model}
\end{figure}

\begin{table}[H]
	\centering
	\begin{tabular}{|l|c|c|c|c|}
		\hline
		&\textbf{no noise}&\textbf{with noise}&\textbf{double noise}\\\hline
		\textbf{nmpc-codegen}&3 ms&12 ms&10 ms \\\hline
		\textbf{draft panoc}&8 ms&18 ms&22 ms \\\hline
		\textbf{OPTI:ipopt}&45 ms&51 ms&80 ms \\\hline
	\end{tabular}
	\caption{average till convergence in miliseconds}
	\label{tbl:average till convergence noise}
\end{table}

% only do simulations with trailer model, quadcopter models goes nuts
% 2 figures , one with noise and one without use draft PANOC with ipopt
%TODO add simulation with noise

\section{Problem with local minimum}
One problem with the approach discussed in this chapter, it's visible in the third demo. The results of the third demo are illustrated in figure~\ref{fig:demo: local minimum problem}, the trailer drives through the obstacle which is obviously impossible in reality. In reality the trailer will simply crash into the obstacle, and get stuck. Of course it is possible to increase the weight on the obstacle, this will make the problem harder. As the condition will get worse, and the solution will be about the same. The trainer will get stuck against the obstacle.

\begin{figure}[H]
	\centering
	\includegraphics[width=0.5\textwidth]{demos/demo3}
	\caption{Local minimum problem}
	\label{fig:demo: local minimum problem}
\end{figure}

\section{Conclusion}

The simulations clearly indicate that a great speed up his gained by implementing the panoc algorithm in C. As the algorithm doesn't require a lot of memory it can be run efficiently on small embedded devices. The control engineer can design and test the performance of the algorithm in either Matlab or Python.


\chapter{Code analysis of panoc}



\begin{figure}[H]
	\centering
	\includegraphics[width=0.5\textwidth]{intel_analysis/solution}
	\caption{solution nmpc problem analysis}
	\label{fig:solution nmpc problem analysis}
\end{figure}

\section{Intel Profiler}

\begin{figure}[H]
	\centering
	\begin{subfigure}[b]{0.45\textwidth}
		\centering
		\includegraphics[width=1.2\textwidth]{intel_analysis/no_mkl}
		\caption{no mkl}
		\label{fig:hotspot no mkl}
	\end{subfigure}
	\hfill
	\begin{subfigure}[b]{0.45\textwidth}
		\centering
		\includegraphics[width=1.2\textwidth]{intel_analysis/with_mkl}
		\caption{with mkl}
		\label{fig:hotspot with mkl}
	\end{subfigure}
	\caption{Hot spot analysis with Intel Vtune Amplifier 2018}
\end{figure}

\begin{figure}[H]
	\centering
	\includegraphics[width=1\textwidth]{intel_analysis/no_mkl_topdown}
	\caption{solution nmpc problem analysis}
	\label{fig:no mkl top down}
\end{figure}
\begin{figure}[H]
	\centering
	\includegraphics[width=1\textwidth]{intel_analysis/with_mkl_top_down}
	\caption{solution nmpc problem analysis}
	\label{fig:with mkl top down}
\end{figure}


%\csvstyle{myTableStyle}{tabular=|c|c|}

\begin{center}
	\begin{tabular}{| l | c |}
		\hline
		Function&CPU Time: Total \\
		\hline
		casadi\_interface\_f\_df & 44.30\%\\
		lbfgs\_get\_direction & 43.80\% \\
		proximal\_gradient\_descent\_forward\_backward\_envelop & 37.00\%\\
		buffer\_evaluate\_new\_location & 32.40\%\\
		inner\_product & 26.20\%\\
		libm\_sse2\_sincos & 18.60\%\\
		vector\_add\_ntimes & 13.00\%\\
		proximal\_gradient\_descent\_get\_direction & 11.90\%\\
		\hline
	\end{tabular}
	\captionof{table}{Table Title}
\end{center}

\begin{center}
	\begin{tabular}{| l | c |}
		\hline
		Function&CPU Time: Total \\
		\hline
		casadi\_interface\_f\_df&55.30\% \\
		proximal\_gradient\_descent\_forward\_backward\_envelop&49.70\% \\
		buffer\_evaluate\_new\_location&43.50\% \\
		lbfgs\_get\_direction&38.50\% \\
		vector\_add\_ntimes&35.90\% \\
		proximal\_gradient\_descent\_get\_direction&11.80\% \\
		libm\_sse2\_sincos&11.80\% \\
		inner\_product&8.80\% \\
		\hline
	\end{tabular}
	\captionof{table}{Table Title}
\end{center}
\section{Valgrind callgrind}

\begin{figure}
	\centering
	\includegraphics[width=1\textwidth]{callgrid/graph}
	\caption{callgrid}
	\label{fig:callgrid}
\end{figure}
\chapter{The Augmented Lagrangian}
Until now there are two ways to add constraints to the control problem: use a proximal function or add a soft constraint to the cost function. It is up to the user to determine the weights of the soft constraints. The weights that are too high, will result in a badly conditioned problem. Putting the weights too low will result more violations of the soft constraints.

\section{Definition of the augmented Lagrangian}
	One way to reduce the ill conditioning of the problem when increasing the weights of the soft constraints, is the usage of the augmented Lagrangian as described in Chapter 17 of \cite{Wright}. Furthermore current research done by Ben Hermans at the KuLeuven suggests a way to update the weights or also called the penalty parameters.
	
	\begin{equation}
		\begin{aligned}
			& \underset{x}{\text{argmin}}
			& & f_0(x) \\
			& \text{subject to}
			& & g(x)=0
		\end{aligned}
		\label{eq:opti problem lagrangian}
	\end{equation}
	
	The problem described in equation~\ref{eq:opti problem lagrangian} can be solved by minimizing equation~\ref{eq:lagrangian quadratic model}  for x with a fixed $\mu_k$, and then solving again for x with $mu_{k+1}>\mu_k $. As mentioned before increasing $\mu$ will worsen the condition of the problem. 
	
	\begin{equation}
		Q(x,\mu) \overset{def}{=} f(x) + \mu \sum_{i \in \epsilon} c_i^2(x)
		\label{eq:lagrangian quadratic model}
	\end{equation}
	
	By adding Lagrangian multipliers equation~\ref{eq:augmented lagrangian definition} is obtained.However the feasibility condition $c_i(x)=0$ will not be met in the first iterations. As with each time the problem solved, the condition of equation~\ref{eq:perturbed feasibility conditions} is met. As $c_i(x)$ goes to zero and $\mu_k$ goes to infinity the condition of equation~\ref{eq:perturbed feasibility conditions} will get closer to that of the feasibility condition.
	
	\begin{equation}
		\Lagr_A(x,\lambda;\mu) \overset{def}{=} f(x) - \sum_{i \in \epsilon}\lambda_ic_i(x) + \mu \sum_{i \in \epsilon}c_i^2(x)
		\label{eq:augmented lagrangian definition}
	\end{equation}
			
	\begin{equation}
		c_i(x_k) = -\lambda_i/\mu_i
		\label{eq:perturbed feasibility conditions}
	\end{equation}
	

\section{Optimality conditions}
	The value of the Lagrangian multiplier can be determined from equation~\ref{eq:gradient augmented lagrangian definition}. The Lagrangian multipliers are determined trough the optimality condition. Which can be obtained by taking the derivative of the Lagrangian towards x, and putting it to zero.($\nabla_x \Lagr_A(x_k,\lambda_k;\mu_k) = 0$) The gradient will only be zero, if the optimality conditions displayed in equation~\ref{eq:optimality codition augmented lagrangian definition} are met.
	\begin{equation}
		\lambda_{k+1}^{i} = \lambda_{k}^{i} - 2\mu_k c_i(x_k)
		\label{eq:optimality codition augmented lagrangian definition}
	\end{equation}

	\begin{equation}
		\nabla_x \Lagr_A(x_k,\lambda_k;\mu_k) = \nabla f(x_k) - \sum_{i \in \epsilon} [\lambda_i^k - 2\mu_k c_i(x_k)] \nabla c_i(x_k)
		\label{eq:gradient augmented lagrangian definition}
	\end{equation}	
	
	%\begin{equation}
	%	c_i(x) \approx \frac{1}{\mu_k}(\lambda_{k+1} - \lambda_k)
	%\end{equation}
	
	
\section{Algorithm}
	The proposed algorithm by cite.. Is algorithm~\ref{alg:panoc with augmented lagrangian}, where $0<\beta<1$. Ben Hermans proposes to take $\mu_{k+1}=\mu_k*2$ if $c_i(x)$is below a certain predefined tolerance. 
	
	\begin{algorithm}
		\caption{panoc nmpc with augmented lagrangian}
		\label{alg:panoc with augmented lagrangian}
		\begin{algorithmic}[1]
			\Procedure {SOLVE\_MPC}{state}
			\State $\mu_0=0$
			\State $\lambda_0=0$
			\While{residual < mpc\_residual }
			\State (residu,input\_horizon) = nmpc\_solve(residual\_solver,state,input\_horizon,$\mu_k$,$\lambda_k$)
			\State $\lambda_{k+1}^{i} = \lambda_{k}^{i} - 2\mu_k c_i(x_k)$
			\State update penality parameter $\mu_{k+1}>\mu_k$
			\State residual\_solver =  residual\_solver$\cdot \beta$
			\EndWhile
			\EndProcedure
		\end{algorithmic}
	\end{algorithm}


	
		
		
\chapter{Conclusion}
%\label{cha:conclusion}
The final chapter contains the overall conclusion. It also contains
suggestions for future work and industrial applications.

The nmpc-codegen library is implemented in Matlab and Python, simple model predictive control problems works very well. As illustrated in the text. It requires absolutely no knowledge of how the solver works. But only requires control theory knowledge to generate the controller. The generate controller, has a very simple interface that requires nearly no control theory knowledge. So that the user of the generated library, does necessarily have to be a control engineer but could be an embedded software engineer.

However as the obstacles get more complex, the problem becomes more complex. And the direct single shot approach results in solutions that are not viable. The algorithm gets stuck on a local minimum, which might be in an nonviable solution.

The last chapter tries to partially solve this by using a Lagrangian. This is ongoing research by Ben Hermans. Its implementation is rather limited as it is ongoing research.

%%% Local Variables: 
%%% mode: latex
%%% TeX-master: "thesis"
%%% End: 


% If you have appendices:
\appendixpage*          % if wanted
\appendix
\section{Function Definitions}
	\subsection{box function}
		1 dimension:
		\begin{equation}
			box_{1D}(u) =
			\begin{aligned}
			 & 1 && u \in [-U_b:U_b]\\
			 & 0 && otherwise
			\end{aligned}
			\label{eq:box function 1 dimension}
		\end{equation}
		N dimensions:
		\begin{equation}
		box(u) = min\left[ \sum_{k=1}^ N box_{1D}(u) \right]
		\label{eq:box function N dimensions}
		\end{equation}
	\subsection{Indicator Box function}
		\begin{equation}
			I[box]=
			\begin{aligned}
			& 0 && u \in [-U_b:U_b]\\
			& \inf && otherwise
			\end{aligned}
		\end{equation}
		
		\begin{equation}
		prox[I[box]]=
		\begin{aligned}
		& u && u \in [-U_b:U_b]\\
		& -U_b && u \in [-\inf:-U_b]\\
		& -U_b && u \in [U_b:\inf]\\
		\end{aligned}
		\end{equation}
\section{functions}

	\subsection{Conjugate of strongly convex function}
		Following lemma's are useful from 
		\begin{equation}
			f^*(x)= \underset{u \in dom(f)}{<y,x>-f(y)}
		\end{equation}
		
		If $\nabla f^*$ is lipschitz and obeys \eqref{eq:appendix f lip}, then $f^*$ is well defined and differentiable. (assume dom(f) is convex and closed)
		\begin{equation}
			\nabla f^*(x) = y^* = \argmax <y,x> - f(y)
		\end{equation}
		
		\begin{equation}
			|| \nabla f^*(x) - \nabla f^*(y) ||_2 \leq \mu^{-1} ||x-y||_2
			\label{eq:appendix f lip}
		\end{equation}
\clearpage
\section{Proof FBE alternate equation}

\begin{proof}
	$\varphi_{\gamma} =   f(x) + \underset{y}{\inf} \Big\{ \nabla f(x)^T(y-x) + g(y) + \frac{1}{2 \gamma} ||x-y||^2  \Big\} $
	\begin{align*}
	g^{\gamma} 	&=  \underset{y}{\inf} \big \{f(y)+\frac{1}{2 \cdot \gamma}||x-y||^2 \big \} \\
	\varphi_{\gamma} 
	&= f(x) - \frac{\gamma}{2}||\nabla f(x)||^2 + g^{\gamma} \big(x-\gamma \nabla f(x) \big) \\
	&= f(x) - \frac{\gamma}{2}||\nabla f(x)||^2 + g^{\gamma} \big(\bar{x} \big)\\
	g^{\gamma} (\bar{x})
	&=\underset{y}{\inf} \Big\{g(y)+\frac{1}{2 \gamma}||\bar{x}-y||^2 \Big\}	\\
	\frac{1}{2 \gamma}||\bar{x}-y||^2
	&=\frac{1}{2 \gamma} \Big [ (\bar{x}-y)^T(\bar{x}-y) \Big]\\
	&=\frac{1}{2 \gamma} \Big [ x^Tx - 2 x^Ty + y^Ty \Big]\\
	\bar{x}^T\bar{x}
	&=[x- \gamma \nabla f(x)]^T[x- \gamma \nabla f(x)] \\
	&= x^Tx -2x^T\nabla f(x) \gamma + \gamma^2 \nabla f(x)^T\nabla f(x)-2x^Ty \\
	&=-2(x-\gamma f\nabla(x))^Ty\\
	&=-2x^Ty + 2\gamma \nabla f(x)^Ty \\
	\frac{1}{2 \gamma}||\bar{x}-y||^2 
	& =\frac{1}{2 \gamma}[x^Tx-2x^T\nabla f(x) \gamma + \gamma^2 \nabla f(x)^T\nabla f(x) -2x^Ty + 2\gamma \nabla f(x)^Ty +y^Ty] \\
	& = \frac{1}{2 \gamma}[-2x^T\nabla f(x) \gamma  + 2\gamma \nabla f(x)^Ty + \gamma^2 \nabla f(x)^T\nabla f(x) +x^Tx -2x^Ty +y^Ty]\\
	&= \frac{1}{2 \gamma}[ 2\gamma \nabla f(x)^T(y-x) + \gamma^2||\nabla f(x)||^2 + (x-y)^T(x-y)]\\
	&= \frac{1}{2 \gamma}[ 2\gamma \nabla f(x)^T(y-x) + \gamma^2||\nabla f(x)||^2 + ||x-y||^2] \\
	&=  \nabla f(x)^T(y-x) +\frac{\gamma}{2}||\nabla f(x)||^2 + \frac{1}{2 \gamma} ||x-y||^2 \\
	g^{\gamma} (\bar{x})
	&=\underset{y}{\inf} \Big\{g(y)+ \nabla f(x)^T(y-x) +\frac{\gamma}{2}||\nabla f(x)||^2 + \frac{1}{2 \gamma} ||x-y||^2  \Big\} \\
	&= \frac{\gamma}{2}||\nabla f(x)||^2 + \underset{y}{\inf} \Big\{g(y)+ \nabla f(x)^T(y-x) + \frac{1}{2 \gamma} ||x-y||^2  \Big\} \\
	\varphi_{\gamma} 
	&= f(x) - \frac{\gamma}{2}||\nabla f(x)||^2 + g^{\gamma} \big(\bar{x} \big)\\
	&= f(x) - \frac{\gamma}{2}||\nabla f(x)||^2 +  \frac{\gamma}{2}||\nabla f(x)||^2 + \underset{y}{\inf} \Big\{g(y)+ \nabla f(x)^T(y-x) + \frac{1}{2 \gamma} ||x-y||^2  \Big\}\\
	&=   f(x) + \underset{y}{\inf} \Big\{ \nabla f(x)^T(y-x) + g(y) + \frac{1}{2 \gamma} ||x-y||^2  \Big\} 
	\end{align*}
	\label{prf:}
\end{proof}



$$  $$
$$  $$



$$ $$

\backmatter
% The bibliography comes after the appendices.
% You can replace the standard "abbrv" bibliography style by another one.
\bibliographystyle{abbrv}
\bibliography{references}

\end{document}