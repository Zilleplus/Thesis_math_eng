\documentclass[master=eelt,masteroption=ei]{kulemt}
\setup{title={The best master's thesis ever},
  author={First Author\and Second Author},
  promotor={Prof.\,dr.\,ir.\ Knows Better},
  assessor={Ir.\,Kn. Owsmuch\and K. Nowsrest},
  assistant={Ir.\ An~Assistent \and A.~Friend}}
% The following \setup may be removed entirely if no filing card is wanted
\setup{filingcard,
  translatedtitle=,
  udc=621.3,
  shortabstract={Here comes a very short abstract, containing no more than 500
    words. \LaTeX\ commands can be used here. Blank lines (or the command
    \texttt{\string\pa r}) are not allowed!
    \endgraf \lipsum[2]}}
% Uncomment the next line for generating the cover page
%\setup{coverpageonly}
% Uncomment the next \setup to generate only the first pages (e.g., if you
% are a Word user.
%\setup{frontpagesonly}

% Choose the main text font (e.g., Latin Modern)
\setup{font=lm}

% If you want to include other LaTeX packages, do it here. 

% Finally the hyperref package is used for pdf files.
% This can be commented out for printed versions.
\usepackage[pdfusetitle,colorlinks,plainpages=false]{hyperref}

%%%%%%%
% The lipsum package is used to generate random text.
% You never need this in a real master's thesis text!
\IfFileExists{lipsum.sty}%
 {\usepackage{lipsum}\setlipsumdefault{11-13}}%
 {\newcommand{\lipsum}[1][11-13]{\par And some text: lipsum ##1.\par}}
%%%%%%%

%\includeonly{chap-n}
\begin{document}

\begin{preface}
  I would like to thank everybody who kept me busy the last year,
  especially my promoter and my assistants. I would also like to thank the
  jury for reading the text. My sincere gratitude also goes to my wive and
  the rest of my family.
\end{preface}

\tableofcontents*

\begin{abstract}
  The \texttt{abstract} environment contains a more extensive overview of
  the work. But it should be limited to one page.

  \lipsum[1]
\end{abstract}

% A list of figures and tables is optional
%\listoffigures
%\listoftables
% If you only have a few figures and tables you can use the following instead
\listoffiguresandtables
% The list of symbols is also optional.
% This list must be created manually, e.g., as follows:
\chapter{List of Abbreviations and Symbols}
\section*{Abbreviations}
\begin{flushleft}
  \renewcommand{\arraystretch}{1.1}
  \begin{tabularx}{\textwidth}{@{}p{12mm}X@{}}
    LoG   & Laplacian-of-Gaussian \\
    MSE   & Mean Square error \\
    PSNR  & Peak Signal-to-Noise ratio \\
  \end{tabularx}
\end{flushleft}
\section*{Symbols}
\begin{flushleft}
  \renewcommand{\arraystretch}{1.1}
  \begin{tabularx}{\textwidth}{@{}p{12mm}X@{}}
    42    & ``The Answer to the Ultimate Question of Life, the Universe,
            and Everything'' according to \cite{h2g2} \\
    $c$   & Speed of light \\
    $E$   & Energy \\
    $m$   & Mass \\
    $\pi$ & The number pi \\
  \end{tabularx}
\end{flushleft}

% Now comes the main text
\mainmatter

\chapter{Introduction}
\label{cha:intro}
The first contains a general introduction to the work. The goals are
defined and the modus operandi is explained.

\section{Lorem Ipsum 4--5}
\lipsum[4-5]

\section{Lorem Ipsum 6--7}
\lipsum[6-7]

%%% Local Variables: 
%%% mode: latex
%%% TeX-master: "thesis"
%%% End: 

\chapter{The First Chapter}
\label{cha:1}
A chapter is a logical unit. It normally starts with an introduction, which
you are reading now. The last topic of the chapter holds the conclusion.

\section{The First Topic of the Chapter}
First comes the introduction to this topic.

\lipsum[55]


\chapter{Conclusion}
\label{cha:conclusion}

\section{Nmpc-codegen library}
The controller code generation and simulation is available in both Python and Matlab, which are computer languages with a fast learning curve. It is very straightforward for a control engineer to design and test a controller, without knowing anything about low level programming.

The generated controller, has a simple interface that requires nearly no control theory knowledge. So that the user of the generated C library doesn't necessarily has to be a control engineer but could be an embedded software engineer. Who has no idea how a MPC controller works, but knows how embedded systems work.

However as the MPC problem becomes more complex. The algorithm can get stuck on a local minimums, which might not be a viable solution. It might violate the constraint by a too large margin. This can partially be avoided by properly tuning the weights of the constraints. 

This weight tuning is rather hard, as increasing the weights worsens the condition of the problem. So the controller designer often ends up with an impossible choice. The last chapter partially solves this by using a Lagrangian. It is easier to tune the parameters and it improves the performance of the controller. 

\section{Future work}
The PANOC algorithm does not require advanced algebra operations such as solving a system. Only simple vector additions and inner product are required. So it's straightforward to implement PANOC with flip-flops and adders. The only hard part is the cost function and it's gradient, as Casadi does not support VHDL or Verilog at this time.

\section{Industrial applications}
Embedded model predictive control is used in a wide variety of domains, here are a few examples:
\begin{itemize}
	\item Unmanned Aerial vehicles (UAV's)
	\item Self driving robots
	\item Anti-lock brakes cars
	\item Adaptive cruise control cars
	\item Robotics
\end{itemize}

%%% Local Variables: 
%%% mode: latex
%%% TeX-master: "thesis"
%%% End: 

% If you have appendices:
\appendixpage*          % if wanted
\appendix
\chapter{The First Appendix}


%%% Local Variables: 
%%% mode: latex
%%% TeX-master: "thesis"
%%% End: 



\backmatter
% The bibliography comes after the appendices.
% You can replace the standard "abbrv" bibliography style by another one.
\bibliographystyle{abbrv}
\bibliography{references}

\end{document}

%%% Local Variables: 
%%% mode: latex
%%% TeX-master: t
%%% End: 
