\chapter{Proximal gradient method}
	\section{Proximal mapping}		
		\eqref{eq:indicator function} Is defined as the indicator function of the set C.
		\begin{equation}
		I_c = 
		\begin{cases}
		0 & x \in C  \\
		\infty & x \notin C
		\end{cases}
		\label{eq:indicator function}
		\end{equation}
		
		The proximal mapping in \cite{NealParikh} is defined as $\prox_{\gamma g}(x)= \underset{u}{\argmin}(g(u) + \frac{1}{2 \gamma}||u-x||^2_2)$. 
		
		2 simple examples of the proximal mappings are:
		\begin{enumerate}
			\item $h(x)=0$ then $\prox_{\gamma h}(x)=x$ 
			\item $h(x)=I_C(x)$ where $I_c$ is \eqref{eq:indicator function}, $\prox_{\gamma h}(x)$ is the orthogonal projection of $x$ onto the set $C$.
		\end{enumerate}

		The proximal mapping can be seen as a generalization of a projection. Appendix~\ref{appendix:indicator box function} contains an other important example, the indicator box function.
	
	\section{Gradient projected method}
		
		\begin{equation}
			\begin{aligned}
			& \underset{x}{\text{argmin}}
			& & f_0(x) \\
			& \text{subject to}
			& & g(x)=0
			\end{aligned}
			\label{eq:prox grad opti problem}
		\end{equation}
		
		The classical gradient decent method cannot be used to solve the problem of \eqref{eq:prox grad opti problem}. Because this problem has a condition $g(x)=0$ that must be met by the solution. If in each iteration the solution is projected on the space spanned by $g(x)=0$, the iteration of \eqref{eq:grad descent} becomes \eqref{eq:projected grad descent}. This algorithm is called the gradient projection method, and every iteration now complies with the condition $g(x)=0$
		
		\begin{equation}
			x^k = x^{k-1} - \gamma \nabla f(x^{k-1})
			\label{eq:grad descent}
		\end{equation}
		
		\begin{equation}
			x^k = \project_{g(x)=0}[ x^{k-1} - \gamma \nabla f(x^{k-1})]
			\label{eq:projected grad descent}
		\end{equation}
		
		If $g(x)$ is the indicator function of the set onto which was projected. Then  \eqref{eq:projected grad descent} can be written as \eqref{eq:proximal grad descent}, known as the proximal gradient method.
		
		\begin{equation}
				x^k = \prox_{\gamma g}[ x^{k-1} - \gamma \nabla f(x^{k-1})]
			\label{eq:proximal grad descent}
		\end{equation}
	
	\section{The proximal gradient method}
		\eqref{eq:prox grad problem} Can be solved with the proximal gradient method, if the proximal mapping of $g(x)$ is analytically defined.  The proximal gradient method is sometimes called forward backward splitting (FBS). Moving down with the gradient is the forward movement, and the proximal mapping is the backward movement.
		
			\begin{equation}
			\underset{x}{\argmin} = f(x) + g(x)
			\label{eq:prox grad problem}
			\end{equation}
		
		Inspired by the projected gradient method, the proximal gradient method is defined as \eqref{eq:prox grad method} in \cite{NealParikh}. The $\gamma$ variable is the step size, in order to have a convergence of O(1/k) $\gamma \in(0,1/L)$. If $\gamma \in (1/L,2/L)$ the algorithm will still converge but then its no longer a majorization-minimization method. (more on this see \cite{NealParikh})
		
		\begin{equation}
			x^k = \prox_{\gamma g}\big( x^{(k-1)}- \gamma \nabla f(x^{(k-1)})\big)
			\label{eq:prox grad method}
		\end{equation}	
	
	\section{Proximal minimization algorithm}
		 \cite{QianYang} Contains a short proof that illustrates the proximal mapping as a fixed point minimization algorithm. A special property of the conjugated function is used to derive the gradient, see appendix~\ref{appendix:conjugate of strongly convex function} for the theorem. This means thats the iteration $x^{k+1}=\prox_g(x^k)$ will minimize the smoothed version of function g(x). 
		 
		 \begin{proof}
		 	The iteration $x^{k+1}=\prox_g(x^k)$ will minimize the smoothed version of function f(x). The original proof can be found in \cite{QianYang}. 
		 	\begin{align*}
		 	f_{\mu}
		 	&= \underset{y}{\inf}\Big\{ |y| +\frac{1}{2 \mu}(x-y)^2 \Big\} \\
		 	&=   \frac{1}{2 \mu}||x||^2 + \frac{1}{2\mu} 
		 	\underset{y}{\inf}\Big\{
		 	2 \mu f(y) - 2x^T y + ||y||^2
		 	\Big\} \\
		 	&=  \frac{1}{2 \mu}||x||^2 + \frac{1}{\mu} 
		 	\underset{y}{\sup}\Big\{
		 	x^T y  - \mu f(y) - \frac{1}{2} ||y||^2 \Big \} \\
		 	&= \frac{1}{2\mu }||x||^2 \frac{1}{\mu } \Big( \mu f + \frac{1}{2}||\cdot||^2 \Big)^* (x) \\
		 	\nabla  f_{\mu} 
		 	&= \frac{x}{\mu} - \frac{1}{\mu} \underset{y}{\argmax} 
		 	\Big \{ x^Ty - \mu f(y) - \frac{1}{2}||y||^2 \Big \}\\
		 	& = \frac{1}{\mu}(x - \prox_{\mu f}(x)) \\
		 	\prox_{\mu f}(x)
		 	& = x- \mu \nabla f_{\mu}(x)
		 	\end{align*}
		 	\label{prf:proximal minimiztion alg proof}
		 \end{proof}	 

\section{Proximal gradient method with line search}
	\subsection{Starting value gamma }
		\subsubsection{Estimating Lipschitz value}
			The Lipschitz of $\nabla f(x)$ value is a non negative number that complies with \eqref{eq:definition lipschitz value}.
			\begin{equation}
			L = \underset{x \neq y}{\sup} \frac{|\nabla f(y)-\nabla f(x)|}{|y-x|}
			\label{eq:definition lipschitz value}
			\end{equation}
			
			In practice it is not always possible to find the actual Lipschitz value. So the Lipschitz value is estimated ($L'$) locally at the starting location $x_0$ using \eqref{eq:estimated lipschitz value in starting position}. With $\delta=max[\delta_l,10^{-6} \cdot x_0]$ where $\delta_l$ is a small number chosen by the controller designer.
			% in code the safety value is DELTA_LIPSCHITZ_SAFETY_VALUE and set to 10^-6
			
			\begin{equation}
			L' = \frac{|\nabla f(x_0+\delta)-\nabla f(x_0)|}{|\delta|}
			\label{eq:estimated lipschitz value in starting position}
			\end{equation}
			
			The Lipschitz value is not explicitly saved but is used to estimate the line-search parameter $\gamma$. The backtracking of the proximal gradient algorithm then further improves $\gamma$. As the algorithm progresses and $\gamma$ improves, so does the estimation of the Lipschitz value indirectly.
		
		\subsubsection{Estimating gamma}	
			\cite{LorenzoStella2017} states that $\gamma<\frac{1}{L}$ guarantees convergence to a local minimum. As gamma needs to be smaller than $\frac{1}{L}$ a safety value is introduced. This idea was copied over from the kul-forbes/ForBES library by Lorenzo Stella and Panos Patrinos, which uses a $\beta$ of 0.05. And leads to \eqref{eq:starting value gamma}.
			\begin{equation}
			\gamma = \frac{1-\beta}{L}
			\label{eq:starting value gamma}
			\end{equation}		
	
	\subsection{Backtracking gamma}			
		\subsubsection{Backtracking in proximal gradient descent used in FBS}
			The line-search makes use of the quadratic bound of \eqref{eq:quadratic bound linesearch prox grad} Which is an quadratic bound, by backtracking on $\gamma$ the optimal step size is retrieved.
			
			\begin{equation}
			f({\bar{x}}) \leq f(x) - \nabla f(x)^T[x-\bar{x}] + \frac{1}{2 \gamma}||x-\bar{x}||^2
			\label{eq:quadratic bound linesearch prox grad}
			\end{equation}
			
			This leads to algorithm~\ref{alg:backtracking on gamma} the proximal gradient method. \eqref{eq:quadratic bound linesearch prox grad} can be seen as a quadratic model, as the Lipschitz value of the gradient is equal to $L=\frac{1}{\gamma}$.
			
			\begin{algorithm}
				\caption{backtracking $\gamma$}
				\label{alg:backtracking on gamma}
				\begin{algorithmic}[1]
					\Procedure {lineasearch\_gamma}{x,$\gamma$}
					\State $\bar{x} = \prox_{\gamma g}\big( x- \gamma \nabla f(x)\big)$
					\While{$f({\bar{x}}) > f(x) - \nabla f(x)^T[x-\bar{x}] + \frac{1}{2 \gamma}||x-\bar{x}||^2$}
					\State $\gamma = \frac{\gamma}{2}$
					\State $\bar{x} = \prox_{\gamma g}\big( x- \gamma \nabla f(x)\big)$
					\EndWhile
					\State \Return $\gamma$
					\EndProcedure
				\end{algorithmic}
			\end{algorithm}
	\subsection{Final algorthm}
		The final algorithm~\ref{alg:proximal gradient PANOC with backtracking} delivers the upward direction. $x_{k+1}=x_k - direction$.
		\begin{algorithm}
			\caption{proximal gradient PANOC with backtracking}
			\label{alg:proximal gradient PANOC with backtracking}
			\begin{algorithmic}[1]
				\Procedure{get\_proximal\_gradient\_step}{x,$\gamma$}
				\State $\gamma$=LINESEARCH\_GAMMA($\gamma$)
				\State $\bar{x} = \prox_{\gamma g}\big( x- \gamma \nabla f(x)\big)$
				\State \Return direction=$[x-\bar{x}]$, $\gamma$
				\EndProcedure
			\end{algorithmic}
		\end{algorithm}
\section{Proximal gradient alternative view}
	\subsection{Majorization-minimization algorithm}
	A Majorization-minimization algorithm is a type of algorithm that minimizes a surrogate function. The surrogate function is a approximation of the actual problem. And needs to have the conditions described in \eqref{eq:MM algorithm conditions}, where h(x,y) is a surrogate function and f(x) is the actual problem. The problem described in \eqref{eq:MM algorithm formula step} is iteratively solved until convergence.
	\begin{equation}
		\begin{cases}
			h(x,x) = f(x) \\
			h(x,y) \leq f(x)
		\end{cases}
		\label{eq:MM algorithm conditions}
	\end{equation}
	
	\begin{equation}
		x_{k+1} = \argmin_y h(x_k,y)
		\label{eq:MM algorithm formula step}
	\end{equation}

	\subsection{The proximal gradient as majorization-minimization algorithm}
	\cite{NealParikh} States that the proximal gradient algorithm can be seen as a majorization-minimization algorithm. Its surrogate function is described in \eqref{eq:surrogate function}, proof~\ref{prf:proximal gradient as MM} illustrates how minimizing the surrogates function is equal as taking a step of the proximal gradient algorithm.
	
	\begin{equation}
		h(x,y) = f(x) + \nabla f(x)^T(x-y) + \frac{1}{2 \cdot \gamma}||x-y||^2_2
		\label{eq:surrogate function}
	\end{equation}
	
	\begin{proof}
		$h(x,y) = f(x) + \nabla f(x)^T(x-y) + \frac{1}{2 \cdot \gamma}||x-y||^2_2$ is the MM surugate function of proximal gradient algorithm. The gradient is set too zero and solved for y.
		\begin{align*}
		h(x,y)
		& = f(x) + \nabla f(x)^T(x-y) + \frac{1}{2 \cdot \gamma}||x-y||^2_2 \\
		\frac{\partial h(x,y)}{\partial y}
		& = 0 + \nabla f(x) + \frac{1}{2 \cdot \gamma} + \frac{1}{2 \cdot \gamma}\frac{\partial ||x-y||^2_2}{\partial y}  \\	
		\frac{\partial ||x-y||^2_2}{\partial y}
		& = \frac{\partial(x-y)^T(x-y)}{\partial y} = \frac{\partial(-2x^Ty + y^Ty)}{\partial y} = -2x+y \\
		\frac{\partial h(x,y)}{\partial y}
		& = \nabla f(x) - \frac{x}{\gamma} + \frac{y}{2 \gamma} = 0 \\
		0
		& = \gamma \nabla f(x) - x + \frac{y}{2} \\
		y & = \gamma \nabla f(x) - x
		\end{align*}
		\label{prf:proximal gradient as MM}
	\end{proof}