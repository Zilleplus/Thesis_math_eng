\documentclass[master=elt,masteroption=eg,english]{kulemt}
\setup{title={The best master's thesis ever},
  author={Een Auteur\and Tweede Auteur},
  promotor={Prof.\,dr.\,ir.\ Weet Beter},
  assessor={Ir.\,W. Eetveel\and W. Eetrest},
  assistant={Ir.\ A.~Assistent \and D.~Vriend}}
% De volgende \setup mag verwijderd worden als geen fiche gewenst is.
\setup{filingcard,
  translatedtitle={Beste masterproef ooit al geschreven},
  udc=621.3,
  shortabstract={Hier komt een heel bondig abstract van hooguit 500
    woorden. \LaTeX\ commando's mogen hier gebruikt worden. Blanco lijnen
    (of het commando \texttt{\string\pa r}) zijn wel niet toegelaten!
    \endgraf \lipsum[2]}}
% Verwijder de "%" op de volgende lijn als je de kaft wil afdrukken
%\setup{coverpageonly}
% Verwijder de "%" op de volgende lijn als je enkel de eerste pagina's wil
% afdrukken en de rest bv. via Word aanmaken.
%\setup{frontpagesonly}

% Kies de fonts voor de gewone tekst, bv. Latin Modern
\setup{font=lm}

% Hier kun je dan nog andere pakketten laden of eigen definities voorzien

% Tenslotte wordt hyperref gebruikt voor pdf bestanden.
% Dit mag verwijderd worden voor de af te drukken versie.
\usepackage[pdfusetitle,colorlinks,plainpages=false]{hyperref}

%%%%%%%
% Om wat tekst te genereren wordt hier het lipsum pakket gebruikt.
% Bij een echte masterproef heb je dit natuurlijk nooit nodig!
\IfFileExists{lipsum.sty}%
 {\usepackage{lipsum}\setlipsumdefault{11-13}}%
 {\newcommand{\lipsum}[1][11-13]{\par Hier komt wat tekst: lipsum ##1.\par}}
%%%%%%%

%\includeonly{chap-n}
\begin{document}

\begin{preface}
  I would like to thank everybody who kept me busy the last year,
  especially my promoter and my assistants. I would also like to thank the
  jury for reading the text. My sincere gratitude also goes to my wive and
  the rest of my family.
\end{preface}

\tableofcontents*

\begin{abstract}
  The \texttt{abstract} environment contains a more extensive overview of
  the work. But it should be limited to one page.

  \lipsum[1]
\end{abstract}

\begin{abstract*}
  In dit \texttt{abstract} environment wordt een al dan niet uitgebreide
  Nederlandse samenvatting van het werk gegeven.
  Wanneer de tekst voor een Nederlandstalige master in het Engels wordt
  geschreven, wordt hier normaal een uitgebreide samenvatting verwacht,
  bijvoorbeeld een tiental bladzijden. 

  \lipsum[1]
\end{abstract*}

% Een lijst van figuren en tabellen is optioneel
%\listoffigures
%\listoftables
% Bij een beperkt aantal figuren en tabellen gebruik je liever het volgende:
\listoffiguresandtables
% De lijst van symbolen is eveneens optioneel.
% Deze lijst moet wel manueel aangemaakt worden, bv. als volgt:
\chapter{List of Abbreviations and Symbols}
\section*{Abbreviations}
\begin{flushleft}
  \renewcommand{\arraystretch}{1.1}
  \begin{tabularx}{\textwidth}{@{}p{12mm}X@{}}
    LoG   & Laplacian-of-Gaussian \\
    MSE   & Mean Square error \\
    PSNR  & Peak Signal-to-Noise ratio \\
  \end{tabularx}
\end{flushleft}
\section*{Symbols}
\begin{flushleft}
  \renewcommand{\arraystretch}{1.1}
  \begin{tabularx}{\textwidth}{@{}p{12mm}X@{}}
    42    & ``The Answer to the Ultimate Question of Life, the Universe,
            and Everything'' according to \cite{h2g2} \\
    $c$   & Speed of light \\
    $E$   & Energy \\
    $m$   & Mass \\
    $\pi$ & The number pi \\
  \end{tabularx}
\end{flushleft}

% Nu begint de eigenlijke tekst
\mainmatter

\chapter{Introduction}
\label{cha:intro}
The first contains a general introduction to the work. The goals are
defined and the modus operandi is explained.

\section{Lorem Ipsum 4--5}
\lipsum[4-5]

\section{Lorem Ipsum 6--7}
\lipsum[6-7]

%%% Local Variables: 
%%% mode: latex
%%% TeX-master: "thesis"
%%% End: 

\chapter{The First Chapter}
\label{cha:1}
A chapter is a logical unit. It normally starts with an introduction, which
you are reading now. The last topic of the chapter holds the conclusion.

\section{The First Topic of the Chapter}
First comes the introduction to this topic.

\lipsum[55]


\include{chap-2}
% ... en zo verder tot
\include{chap-n}
\chapter{Conclusion}
\label{cha:conclusion}

\section{Nmpc-codegen library}
The controller code generation and simulation is available in both Python and Matlab, which are computer languages with a fast learning curve. It is very straightforward for a control engineer to design and test a controller, without knowing anything about low level programming.

The generated controller, has a simple interface that requires nearly no control theory knowledge. So that the user of the generated C library doesn't necessarily has to be a control engineer but could be an embedded software engineer. Who has no idea how a MPC controller works, but knows how embedded systems work.

However as the MPC problem becomes more complex. The algorithm can get stuck on a local minimums, which might not be a viable solution. It might violate the constraint by a too large margin. This can partially be avoided by properly tuning the weights of the constraints. 

This weight tuning is rather hard, as increasing the weights worsens the condition of the problem. So the controller designer often ends up with an impossible choice. The last chapter partially solves this by using a Lagrangian. It is easier to tune the parameters and it improves the performance of the controller. 

\section{Future work}
The PANOC algorithm does not require advanced algebra operations such as solving a system. Only simple vector additions and inner product are required. So it's straightforward to implement PANOC with flip-flops and adders. The only hard part is the cost function and it's gradient, as Casadi does not support VHDL or Verilog at this time.

\section{Industrial applications}
Embedded model predictive control is used in a wide variety of domains, here are a few examples:
\begin{itemize}
	\item Unmanned Aerial vehicles (UAV's)
	\item Self driving robots
	\item Anti-lock brakes cars
	\item Adaptive cruise control cars
	\item Robotics
\end{itemize}

%%% Local Variables: 
%%% mode: latex
%%% TeX-master: "thesis"
%%% End: 

% Indien er bijlagen zijn:
\appendixpage*          % indien gewenst
\appendix
\chapter{The First Appendix}


%%% Local Variables: 
%%% mode: latex
%%% TeX-master: "thesis"
%%% End: 

% ... en zo verder tot
\include{app-n}

\backmatter
% Na de bijlagen plaatst men nog de bibliografie.
% Je kan de  standaard "abbrv" bibliografiestijl vervangen door een andere.
\bibliographystyle{abbrv}
\bibliography{references}

\end{document}

%%% Local Variables: 
%%% mode: latex
%%% TeX-master: t
%%% End: 
