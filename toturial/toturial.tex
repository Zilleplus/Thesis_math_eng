\documentclass[]{article}

\usepackage{amsmath,amsthm,amssymb}
\DeclareMathOperator{\project}{project}
\DeclareMathOperator{\prox}{prox}
\DeclareMathOperator{\argmax}{argmax}
\DeclareMathOperator*{\minimize}{minimize}
\DeclareMathOperator*{\argmin}{arg\,min}

%\usepackage{program}
\usepackage{algorithm} 
\usepackage{algcompatible}
\usepackage{algpseudocode}
\usepackage{listings}
\usepackage{color}
\usepackage{caption}


\definecolor{codegreen}{rgb}{0,0.6,0}
\definecolor{codegray}{rgb}{0.5,0.5,0.5}
\definecolor{codepurple}{rgb}{0.58,0,0.82}
\definecolor{backcolour}{rgb}{0.95,0.95,0.92}

\lstdefinestyle{mystyle}{
	backgroundcolor=\color{backcolour},   
	commentstyle=\color{codegreen},
	keywordstyle=\color{magenta},
	numberstyle=\tiny\color{codegray},
	stringstyle=\color{codepurple},
	basicstyle=\footnotesize,
	breakatwhitespace=false,         
	breaklines=true,                 
	captionpos=b,                    
	keepspaces=true,                 
	numbers=left,                    
	numbersep=5pt,                  
	showspaces=false,                
	showstringspaces=false,
	showtabs=false,                  
	tabsize=2
}


\usepackage[parfill]{parskip} % paragraphs are seperated with an enter space

\usepackage{graphicx}
\graphicspath{{./figs/}}

%opening
\title{A simple introduction to PANOC}
\author{Willem Melis}
\date{\today}

\begin{document}

\maketitle
\begin{abstract}
This is a simple introduction to the nmpc codegen library. It provides a description of the basic functionality of the library.
\end{abstract}

\section{Introduction}
The main goal of the nmpc-codegen library is to generate non-linear model predictive controllers. The user must provide the nonlinear model of the system and defined the stage and terminal cost. The system equations can either be continuous or discrete. If the system equations are continuous, a discretization scheme will be selected.

The system equations are defined in python Casadi syntax. The Casadi functions are similar to the Matlab functions. A short summary of the available functions, will be provided at the end of this document.

The nonlinear non-convex optimization problem is solved using the PANOC algorithm. This algorithm is guaranteed to converge to a critical point, but not necessarily an global optimum. There are still some parameters that need manual tuning, more on this in the section on the nmpc-controller optional features.

The nmpc-codegen library will generate the controller in C89 code. As the library is aimed at embedded developers. However it is possible to test the controller directly from python.

\subsection{Installation}
The nmpc-codegen library most important dependency is Casadi.The rest of the dependencies are fulfilled if the dependencies of casadi are fulfilled.

Clone the nmpc-codegen repository from github, and add the src\_python folder to the Python path.

\subsection{Quick Start}
The nmpc-codegen library is separated into five parts, Cfunctions containing the constraints, controller containing the obstacles and the controller code. Models containing the two different kinds of models. And finally tools containing the two optional features the simulator and the bootstrapper. The five different parts are discussed in the section below, each in its own chapter.

The Python code used in this document can be found in the repo under the name toturial\_nmpc\_codegen.py .

The final library has the interface display in Listing~\ref{lst:interface library}. An initialization and cleanup function and a solve function.

\begin{lstlisting}[caption={interface library},captionpos=b,label={lst:interface library}]
int nmpc_init();
int nmpc_cleanup();
int npmc_solve(const real_t* current_state,real_t* optimal_inputs);
\end{lstlisting}
\section{Model}
\subsection{Setup a model}

The model contains the relationship between the current state and the next state of the system. The user can provide a discrete model of the system that calculates the next state using the current state. The model also contains the constraints on the input. 

These constraints need to have a special mathematical property of having the proximal operator analytically defined. The available constraints are displayed in table~\ref{tab:constraints}, it is possible to add your own constraints more on this later.

\begin{center}
	\begin{tabular}{ |c|c| } 
		\hline
		Math function & Python function \\ 
		\hline
		I[box(x)] & IndicatorBoxFunction(array\_with\_lower\_bounds,array\_with\_upper\_bounds) \\ 
		\hline
	\end{tabular}
	\label{tab:constraints}
\end{center}

The discrete system equations should be in the form $f(x,u)$  where X is the current states and u are the current inputs. This is demonstrated in listing~\ref{lst:discrete model}.

\begin{lstlisting}[caption={Discrete model},captionpos=b,label={lst:discrete model}]
import nmpccodegen.models as models
import nmpccodegen.Cfunctions as cfunctions

step_size = 0.05
horizon = 300
integrator = "RK" # select a Runga-Kutta  integrator

constraint_input = cfunctions.IndicatorBoxFunction([-1, -1], [1, 1])  
model = models.Model(system_equations,
	constraint_input, step_size, number_of_states, \
	number_of_inputs, coordinates_indices)
\end{lstlisting}

In case the system is defined by continuous function, the model can be defined by the class Model\_continious as demonstrated in the listing~\ref{lst:continious model}. The construct has an additional parameter to identified integration scheme, select one of the integrators from table~\ref{tab:integrators}.

\begin{center}
	\begin{tabular}{ll}
		\hline
		key        & full name integrator \\
		\hline
		BS5        & Bogacki-Shampine RK5(4)8    \\
		BuRK65     & Butcher's RK65              \\
		CMR6       & Calvo 6(5)                  \\
		DP5        & Dormand-Prince RK5(4)7      \\
		FE         & Forward Euler               \\
		Fehlberg45 & Fehlberg RK5(4)6            \\
		Heun33     & Heun RK 33                  \\
		Lambert65  & Lambert                     \\
		MTE22      & Minimal Truncation Error 22 \\
		Merson43   & Merson RK4(3)               \\
		Mid22      & Midpoint Runge-Kutta        \\
		NSSP32     & non-SSPRK 32                \\
		NSSP33     & non-SSPRK 33                \\
		PD8        & Prince-Dormand 8(7)         \\
		RK44       & Classical RK4               \\
		SSP104     & SSPRK(10,4)                 \\
		SSP22      & SSPRK 22                    \\
		SSP22star  & SSPRK22star                 \\
		SSP33      & SSPRK 33                    \\
		SSP53      & SSP 53                      \\
		SSP54      & SSP 54                      \\
		SSP63      & SSP 63                      \\
		SSP75      & SSP 75                      \\
		SSP85      & SSP 85                      \\
		SSP95      & SSP 95                      \\
		\hline
	\end{tabular}
	\captionof{table}{available integrators}
	\label{tab:integrators}	
\end{center}

\begin{lstlisting}[caption={Continue model},captionpos=b,label={lst:continious model}]
import nmpccodegen.models as models

step_size = 0.05
horizon = 300
integrator = "RK" # select a Runga-Kutta  integrator

constraint_input = indbox.IndicatorBoxFunction([-1, -1], [1, 1])  
model = models.Model_continious(system_equations,
	constraint_input, step_size, number_of_states, \
	number_of_inputs, coordinates_indices, integrator)
\end{lstlisting}

\subsection{Defining your own constraints}
\subsubsection{Basics}
It is possible to define your own constraint functions. Listing~\ref{lst:c code implementation proximal functions} illustrates how Cfunctions should actually look. Listing~\ref{lst:c function interface} illustrates how the Python side should look. Every constraint must be a proximal function object. A proximal function of object also contains a function that is the proximal result.

\begin{lstlisting}[caption={c code implementation proximal functions},captionpos=b,label={lst:c code implementation proximal functions}]
void casadi_interface_proxg(const real_t* input,real_t* output){
/* ... add your function in here */
}
real_t casadi_interface_g(const real_t* input){
/* ... add your function in here */
}
\end{lstlisting}

\begin{lstlisting}[caption={c function interface},captionpos=b,label={lst:c function interface}]
class Cfunction:
def __init__(self):
raise NotImplementedError

# save the implementation in c to "location"
def generate_c_code(self,location):
raise NotImplementedError
\end{lstlisting}

\begin{lstlisting}[caption={c proximal function interface},captionpos=b,label={lst:c proximal function interface}]
class ProximalFunction(Cfunction):
def __init__(self,prox):
self._prox=prox

@property
def prox(self):
return self._prox
\end{lstlisting}

\subsubsection{Source file operations library}
Obviously the user is free to use its own libraries to generate the C code. However a small set of functions in the form of class is available to the user, which can make things easier.

The class Source\_file\_generator is available under the Cfunctions sub package. Table contains all the functions available to the user. The constructor of the Source\_file\_generator class specifies if it is either a function or proximal of a function. It is important to make the distinction between the two, because the prototype of the function is different. 

In the following two subsections the indicator box function will be implemented. The mathematical definition of the indicator box function can be found in the appendix. A constraint with an indicator box function allows a certain area of inputs. The proximal function of the me get a box function, simply gives the closest valid input.

\begin{table}
	\begin{center}
		\begin{tabular}{| l | l | }
			\hline
			function & purpose \\ 
			\hline
			open() & open source file \\
			close() & close source file \\
			start\_for(iterator\_name,length,indent) & Start a for loop \\
			close\_for(indent) & Close the for loop \\
			write\_line(line,indent) & Write one line of code\\
			write\_define(name,value,indent) & Define a preprocessor constant\\
			write\_comment\_line(self,line,indent) & Write one line of comment\\
			set\_output(output\_index,value,indent) & Set the value of output array\\
			\hline
		\end{tabular}
	\end{center}
	\label{tbl:function Source_file_generator}
	\caption{Table of the available functions in the Source\_file\_generator Loss}
\end{table}

\subsubsection{Example: Index box function}
First an object is constructed with a certain location and a function type. In this case the function type equal to "g" which means that it's an normal function. The proximal function of this function will be of the type "proxg". Before we can write to the source file we must open a stream, at the end this stream must also be closed.

What follows next is the actual implementation of the function. Followed by the closing of the file stream.

\begin{lstlisting}[caption={example generate g function},captionpos=b,label={lst:example generate g function}]
source_file = Source_file_generator(location,"g")
source_file.open()
source_file.start_for("i","MPC_HORIZON",indent=1)

for dimension in range(0,self._dimension):
	source_file.write_line("if(input["+str(dimension)+"]<"\
		+str(self._lower_limits[dimension])\
		+" || input["+str(dimension)+"]>"\
		+str(self._upper_limits[dimension])+ "){"\
		,indent=1)
	source_file.write_line("return LARGE;", indent=2)
	source_file.write_line("}",indent=1)

source_file.write_line("input+="+str(self._dimension)+ ";",2)
source_file.close_for(indent=1)

source_file.write_line("return 0;", indent=1)
source_file.close()
\end{lstlisting}

\subsubsection{Example: Proximal index box function}
In the previous subsection the indicator box function was implemented. In this subsection the proximal of this function will be implemented. This is why the type of the function is now "proxg" and not "g".

\begin{lstlisting}[caption={example generate proxg function},captionpos=b,label={lst:example generate proxg function}]
source_file = Source_file_generator(location,"proxg")
source_file.open()
source_file.start_for("i","MPC_HORIZON",indent=1)

for dimension in range(0,self._dimension):
	source_file.write_line("if(input["\
		+str(dimension)\
		+"]<"+str(self._lower_limits[dimension])\
		+"){",indent=2)
		
	source_file.set_output(dimension,str(self._lower_limits[dimension])\
		,3)
	source_file.write_line("}else if(input[" \
		+ str(dimension) \
		+ "]>" \
		+ str(self._upper_limits[dimension]) \
		+ "){",\
		indent=2)
	source_file.set_output(dimension, \
		str(self._upper_limits[dimension]), 3)
	source_file.write_line("}else{", 2)
	source_file.set_output(dimension, "input[" \
		+ str(dimension) \
		+ "]", 3)
	source_file.write_line("}", 2)

source_file.write_line("input+="+str(self._dimension)+ ";",2)
source_file.write_line("output+="+str(self._dimension)+ ";",2)
source_file.close_for(indent=1)
source_file.close()
\end{lstlisting}
\section{Controller}
First the absolute minimum amount of arguments to create a controller will be discussed. In the following section the optional features will be discussed.
\subsection{Absolute minimum controller}
In order to create an mpc-controller the stage cost must be defined. The different available stage costs are displayed in table~\ref{tab:stage costs}. The left side of the table contains the mathematical function and the right side contains the corresponding python function. 

Listing~\ref{lst:minimum controller} contains a simple example of a minimal controller. In this case the stage costs needs a reference state. The arrays are always numpy arrays. Finally the controller location (more on this in the chapter on bootstrapper) the model and the stage cost are passed on to the control.

The actual code will only be generated when the generate\_code() function is called. This allows the user, to define additional options.

\begin{center}
	\begin{tabular}{ |c|c| } 
		\hline
		Math function & Python function \\ 
		\hline
		$\int[x^TQx+u^TRu]$ & Stage\_cost\_QR(model, Q, R) \\ 
		\hline
		$\int[(x-x_{ref})^TQ(x-x_{ref})+u^TRu]$ & Stage\_cost\_QR(model, Q, R, reference\_state) \\ 
		\hline
	\end{tabular}
	\label{tab:stage costs}
\end{center}

\begin{lstlisting}[caption={simple controller},captionpos=b,label={lst:minimum controller}]
import nmpccodegen.controller as controller

Q = np.diag([1., 100., 1.])
R = np.eye(model.number_of_inputs, model.number_of_inputs) * 1.

reference_state = np.array([2, 0.5, 0])
stage_cost = controller.Stage_cost_QR_reference(model, Q, R, reference_state)

trailer_controller = npc.Nmpc_panoc(trailer_controller_location,\
	model, stage_cost)
\end{lstlisting}

\subsection{Optional features}
The optional features are displayed in table~\ref{tab:controller features}, and demonstrated in listing~\ref{lst:optional features}.

\begin{lstlisting}[caption={Optional features},captionpos=b,label={lst:optional features}]
import nmpccodegen.controller as controller

reference_state = np.array([2, 0.5, 0])
stage_cost = controller.Stage_cost_QR_reference(model, \
	Q, R, reference_state)

trailer_controller = npc.Nmpc_panoc(trailer_controller_location,\
	model, stage_cost)
trailer_controller.horizon = horizon
trailer_controller.integrator_casadi = True
trailer_controller.panoc_max_steps = 100
trailer_controller.add_obstacle(obstacle, obstacle_weight)
\end{lstlisting}

\begin{center}
	\begin{tabular}{ |c|c|c| } 
		\hline
		attribute & default value & possible values \\ 
		\hline
		data\_type & "double precision" & [double precision, single precision]\\ 
		\hline
		number\_of\_steps & 10 & integer\\ 
		\hline
		lbgfs\_buffer\_size &  10 & integer\\ 
		\hline
		panoc\_max\_steps &  10 & integer\\ 
		\hline
		shooting\_mode & "single shot" & [single shot, multiple shot] \\ 
		\hline
		integrator\_casadi &  False & Boolean=[True,False]\\ 
		\hline
	\end{tabular}
	\captionof{table}{Features of the controller} \label{tab:controller features} 
\end{center}


\section{Simulator}
\subsection{Using the Simulator class}
After the controller is successfully generated, it might be useful to run some simulations in order to get an idea of how well the controller works. That way the user can compare  different controllers with each other.

Listing~\ref{lst:simulator example} illustrates a simple example, where the controller is simulated using the simulator. It is necessary to called the initialize function before the simulation is executed. And the cleanup function after the simulation has been executed.

The results of the simulations, are safe in a simulation object. A simulation object contains the optimal input, and the time to get this input. The time is expressed by six parameters, hours, minutes, seconds, milliseconds, microseconds and nanoseconds. The accuracy of the convergence time, depends on the accuracy of the internal time of the operating system. If an accuracy lower than a millisecond is required, then the users should take in account the internal scheduler of the opening system.

\begin{lstlisting}[caption={Simulator example},captionpos=b,label={lst:simulator example}]
import nmpccodegen.tools as tools

# setup a simulator to test
sim = tools.Simulator(trailer_controller)

# init the controller
sim.simulator_init()

initial_state = np.array([0.01, 0., 0.])
state = initial_state
state_history = np.zeros((number_of_states, number_of_steps))

for i in range(1, number_of_steps):
	result_simulation= sim.simulate_nmpc(state)
	print("Step ["+str(i)+"/"+str(number_of_steps)+ \
		"]: The optimal input is: [" \
		+ str(result_simulation.optimal_input[0]) + "," + \
		str(result_simulation.optimal_input[0]) + "]" \
		+ " time=" + result_simulation.time_string)

	state = np.asarray(model.get_next_state(state, \
		result_simulation.optimal_input))
	state_history[:, i] = np.reshape(state[:], number_of_states)

# cleanup the controller
sim.simulator_cleanup()
\end{lstlisting}

\subsection{Calling the interface directly}
The Ctypes library is a simple way to call C  code from within Python code.If you used the bootstrapper  with the property Python\_interface\_enabled set as true that you must still call Cmake  using the command "Cmake .". A library with the extension ".so" or "dll" will be compiled. It isn't this library that we must call the function "simulation\_init()", "simulation\_cleanup()" and "simulate\_nmpc\_panoc.

\begin{lstlisting}[caption={ctype structure return by simulation},captionpos=b,label={lst:ctype structure return by simulation}]
class Panoc_time(ctypes.Structure):
	_fields_ = [("hours", ctypes.c_int),\
		("minutes", ctypes.c_int),\
		("seconds", ctypes.c_int),\
		("milli_seconds", ctypes.c_int),\
		("micro_seconds", ctypes.c_int),\
		("nano_seconds", ctypes.c_int)]
\end{lstlisting}

\begin{lstlisting}[caption={calling c code directly example},captionpos=b,label={lst:calling c code directly example}]

# Unix-like machines:
nmpc_python_interface = ctypes.CDLL(lib_location) 
# Windows machines
nmpc_python_interface = ctypes.windll.LoadLibrary(lib_location) 

array_state = ctypes.c_double * length_state

array_optimal_input = ctypes.c_double * model.number_of_inputs
optimal_input = array_optimal_input()

nmpc_python_interface.simulation_init()

nmpc_python_interface.simulate_nmpc_panoc.restype = 
	ctypes.POINTER(Panoc_time)
convergence_time = nmpc_python_interface.simulate_nmpc_panoc(\
	array_state(*current_state),\
	optimal_input\
	)

nmpc_python_interface.simulation_cleanup()
\end{lstlisting}
\section{Bootstrapper}

Every controller exists out of static and a dynamic code, the static code can be generated with the bootstrapper. The dynamic code is generated by the nmpc\_controller class. The bootstrap has two parameters, the output location of the controller and the name of the controller. 

The optional parameter python\_interface\_enabled can either be enable or disabled the python interface on the controller. This optional parameter is by default true, and should only be set on false if the user will not use the simulator.

The controller name is used as folder name of the controller. The bootstraper will copy the static code from the library into the new folder. If the folder already exists, it will leave it in place. If the static code is already present, it will print a warning to the screen and replace the files. If the folder structure is already in place, it will leave the folders in place.

\begin{lstlisting}[caption={Optional features},captionpos=b,label={lst:bootstrapper}]
import nmpccodegen.tools as tools


output_locationcontroller = "./test_controller_builds" 
trailer_controller_location = output_locationcontroller + "/" + \
	controller_name + "/"

tools.Bootstrapper.bootstrap(output_locationcontroller,\
	controller_name, python_interface_enabled=True)
\end{lstlisting}

\clearpage


	
\clearpage
\appendix 
\section{Function Definitions}
	\subsection{box function}
		1 dimension:
		\begin{equation}
			box_{1D}(u) =
			\begin{aligned}
			 & 1 && u \in [-U_b:U_b]\\
			 & 0 && otherwise
			\end{aligned}
			\label{eq:box function 1 dimension}
		\end{equation}
		N dimensions:
		\begin{equation}
		box(u) = min\left[ \sum_{k=1}^ N box_{1D}(u) \right]
		\label{eq:box function N dimensions}
		\end{equation}
	\subsection{Indicator Box function}
		\begin{equation}
			I[box]=
			\begin{aligned}
			& 0 && u \in [-U_b:U_b]\\
			& \inf && otherwise
			\end{aligned}
		\end{equation}
		
		\begin{equation}
		prox[I[box]]=
		\begin{aligned}
		& u && u \in [-U_b:U_b]\\
		& -U_b && u \in [-\inf:-U_b]\\
		& -U_b && u \in [U_b:\inf]\\
		\end{aligned}
		\end{equation}
\section{Short summary of Casadi functions}

%\newpage
%\bibliography{mylib}
%\bibliographystyle{plain}

\end{document}
