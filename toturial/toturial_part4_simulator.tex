\section{Simulator}
\subsection{Using the Simulator class}
After the controller is successfully generated, it might be useful to run some simulations in order to get an idea of how well the controller works. That way the user can compare  different controllers with each other.

Listing~\ref{lst:simulator example} illustrates a simple example, where the controller is simulated using the simulator. It is necessary to called the initialize function before the simulation is executed. And the cleanup function after the simulation has been executed.

The results of the simulations, are safe in a simulation object. A simulation object contains the optimal input, and the time to get this input. The time is expressed by six parameters, hours, minutes, seconds, milliseconds, microseconds and nanoseconds. The accuracy of the convergence time, depends on the accuracy of the internal time of the operating system. If an accuracy lower than a millisecond is required, then the users should take in account the internal scheduler of the opening system.

\begin{lstlisting}[caption={Simulator example},captionpos=b,label={lst:simulator example}]
import nmpccodegen.tools as tools

# setup a simulator to test
sim = tools.Simulator(trailer_controller)

# init the controller
sim.simulator_init()

initial_state = np.array([0.01, 0., 0.])
state = initial_state
state_history = np.zeros((number_of_states, number_of_steps))

for i in range(1, number_of_steps):
	result_simulation= sim.simulate_nmpc(state)
	print("Step ["+str(i)+"/"+str(number_of_steps)+ \
		"]: The optimal input is: [" \
		+ str(result_simulation.optimal_input[0]) + "," + \
		str(result_simulation.optimal_input[0]) + "]" \
		+ " time=" + result_simulation.time_string)

	state = np.asarray(model.get_next_state(state, \
		result_simulation.optimal_input))
	state_history[:, i] = np.reshape(state[:], number_of_states)

# cleanup the controller
sim.simulator_cleanup()
\end{lstlisting}

\subsection{Calling the interface directly}
The Ctypes library is a simple way to call C  code from within Python code.If you used the bootstrapper  with the property Python\_interface\_enabled set as true that you must still call Cmake  using the command "Cmake .". A library with the extension ".so" or "dll" will be compiled. It isn't this library that we must call the function "simulation\_init()", "simulation\_cleanup()" and "simulate\_nmpc\_panoc.

\begin{lstlisting}[caption={ctype structure return by simulation},captionpos=b,label={lst:ctype structure return by simulation}]
class Panoc_time(ctypes.Structure):
	_fields_ = [("hours", ctypes.c_int),\
		("minutes", ctypes.c_int),\
		("seconds", ctypes.c_int),\
		("milli_seconds", ctypes.c_int),\
		("micro_seconds", ctypes.c_int),\
		("nano_seconds", ctypes.c_int)]
\end{lstlisting}

\begin{lstlisting}[caption={calling c code directly example},captionpos=b,label={lst:calling c code directly example}]

# Unix-like machines:
nmpc_python_interface = ctypes.CDLL(lib_location) 
# Windows machines
nmpc_python_interface = ctypes.windll.LoadLibrary(lib_location) 

array_state = ctypes.c_double * length_state

array_optimal_input = ctypes.c_double * model.number_of_inputs
optimal_input = array_optimal_input()

nmpc_python_interface.simulation_init()

nmpc_python_interface.simulate_nmpc_panoc.restype = 
	ctypes.POINTER(Panoc_time)
convergence_time = nmpc_python_interface.simulate_nmpc_panoc(\
	array_state(*current_state),\
	optimal_input\
	)

nmpc_python_interface.simulation_cleanup()
\end{lstlisting}