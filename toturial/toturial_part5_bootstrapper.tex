\section{Bootstrapper}

Every controller exists out of static and a dynamic code, the static code can be generated with the bootstrapper. The dynamic code is generated by the nmpc\_controller class. The bootstrap has two parameters, the output location of the controller and the name of the controller. 

The optional parameter python\_interface\_enabled can either be enable or disabled the python interface on the controller. This optional parameter is by default true, and should only be set on false if the user will not use the simulator.

The controller name is used as folder name of the controller. The bootstraper will copy the static code from the library into the new folder. If the folder already exists, it will leave it in place. If the static code is already present, it will print a warning to the screen and replace the files. If the folder structure is already in place, it will leave the folders in place.

\begin{lstlisting}[caption={Optional features},captionpos=b,label={lst:bootstrapper}]
import nmpccodegen.tools as tools


output_locationcontroller = "./test_controller_builds" 
trailer_controller_location = output_locationcontroller + "/" + \
	controller_name + "/"

tools.Bootstrapper.bootstrap(output_locationcontroller,\
	controller_name, python_interface_enabled=True)
\end{lstlisting}